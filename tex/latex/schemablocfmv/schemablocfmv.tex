\documentclass[a4paper,10pt]{article} 
\usepackage[utf8]{inputenc}          
\usepackage[T1]{fontenc}             
\usepackage[francais]{babel}         
\usepackage{lmodern}                 
\usepackage{geometry}
\geometry{hmargin=2.5cm,vmargin=2.6cm}
\usepackage{tikz,pgf,pgfplots}
\pgfplotsset{compat=newest}
\usepackage{tikzfmv}
\usepackage{schemablocfmv}
\usepackage{mathfmv}
\usepackage{amsmath}
\usepackage{titling}
\usepackage{verbatim}
\usetikzlibrary{calc}
\usetikzlibrary{shapes.geometric}
\newcommand{\subtitle}[1]{%                          
\posttitle{%                                   
\par\end{center}%
\begin{center}\large#1\end{center}%
\vskip0.5em}%              
}                          
\newcommand{\mparagraph}[1]{\paragraph{#1}\mbox{}\\} 
\title{Extension de \texttt{schemabloc} avec PGF/Tikz}
\subtitle{schemablocfmv version 0.2}
\author{F. M. Vasconcelos}
\date{}
\begin{document}
\maketitle
\begin{abstract}
Ce paquet définit des macros à partir du paquet \texttt{schemabloc}
de R. Papanicola.
\end{abstract}
%%%%%%%%%%%%%%%%%%%%%%%%%%%%%%%%%%%%%%%%%%%%%%%%%%%%%%%%%%%%%%%%%%%%%%%%%%%%%%%%
\section{Paquets nécessaires}
%%%%%%%%%%%%%%%%%%%%%%%%%%%%%%%%%%%%%%%%%%%%%%%%%%%%%%%%%%%%%%%%%%%%%%%%%%%%%%%%
\begin{verbatim}
\usepackage{tikz}
\usepackage{schemabloc}
\usepackage{schemablocfmv}
\end{verbatim} 
%%%%%%%%%%%%%%%%%%%%%%%%%%%%%%%%%%%%%%%%%%%%%%%%%%%%%%%%%%%%%%%%%%%%%%%%%%%%%%%%
\section{Utilisation des macros}
%%%%%%%%%%%%%%%%%%%%%%%%%%%%%%%%%%%%%%%%%%%%%%%%%%%%%%%%%%%%%%%%%%%%%%%%%%%%%%%%

%%%%%%%%%%%%%%%%%%%%%%%%%%%%%%%%%%%%%%%%%%%%%%%%%%%%%%%%%%%%%%%%%%%%%%%%%%%%%%%
\subsection{Bloc simple}
%%%%%%%%%%%%%%%%%%%%%%%%%%%%%%%%%%%%%%%%%%%%%%%%%%%%%%%%%%%%%%%%%%%%%%%%%%%%%%%
Commande \textbf{b}loc \textbf{s}imple: 

\verb?\bs?
\begin{center}
    \begin{tikzpicture}
        \bs
    \end{tikzpicture}
\end{center}
\hrule
\vspace{0.5cm}

Commande \textbf{b}loc \textbf{s}imple (étendue) : 

\verb?\bs[a][b][c][d]?
\begin{center}
    \begin{tikzpicture}
        \bs[a][b][c][d]
    \end{tikzpicture}
\end{center}
\hrule
\vspace{0.5cm}
\clearpage
%%%%%%%%%%%%%%%%%%%%%%%%%%%%%%%%%%%%%%%%%%%%%%%%%%%%%%%%%%%%%%%%%%%%%%%%%%%%%%%
\subsection{Comparateur}
%%%%%%%%%%%%%%%%%%%%%%%%%%%%%%%%%%%%%%%%%%%%%%%%%%%%%%%%%%%%%%%%%%%%%%%%%%%%%%%
Commande \textbf{comp}arateur + \textbf{b}loc \textbf{s}imple : 

\verb?\compbs?
\begin{center}
    \begin{tikzpicture}
        \compbs
    \end{tikzpicture}
\end{center}
\hrule
\vspace{0.5cm}

Commande \textbf{comp}arateur + \textbf{b}loc \textbf{s}imple (étendue) : 

\verb?\compbs[a][b][c][d][e][f]?
\begin{center}
    \begin{tikzpicture}
        \compbs[a][b][c][d][e][f]
    \end{tikzpicture}
\end{center}
\hrule
\vspace{0.5cm}
%%%%%%%%%%%%%%%%%%%%%%%%%%%%%%%%%%%%%%%%%%%%%%%%%%%%%%%%%%%%%%%%%%%%%%%%%%%%%%%
\subsection{Sommateur}
%%%%%%%%%%%%%%%%%%%%%%%%%%%%%%%%%%%%%%%%%%%%%%%%%%%%%%%%%%%%%%%%%%%%%%%%%%%%%%%
Commande \textbf{som}ateur + \textbf{b}loc \textbf{s}imple : 

\verb?\sombs?
\begin{center}
    \begin{tikzpicture}
        \sombs
    \end{tikzpicture}
\end{center}
\hrule
\vspace{0.5cm}

Commande \textbf{som}arateur + \textbf{b}loc \textbf{s}imple (étendue) : 

\verb?\sombs[a][b][c][d][e][f]?
\begin{center}
    \begin{tikzpicture}
        \sombs[a][b][c][d][e][f]
    \end{tikzpicture}
\end{center}
\hrule
\vspace{0.5cm}
\clearpage
%%%%%%%%%%%%%%%%%%%%%%%%%%%%%%%%%%%%%%%%%%%%%%%%%%%%%%%%%%%%%%%%%%%%%%%%%%%%%%%%
\subsection{Boucle retour unitaire}
%%%%%%%%%%%%%%%%%%%%%%%%%%%%%%%%%%%%%%%%%%%%%%%%%%%%%%%%%%%%%%%%%%%%%%%%%%%%%%%%
Commande \textbf{b}loc + \textbf{r}etour \textbf{uni}taire: 

\verb?\bruni?
\begin{center}
    \begin{tikzpicture}
        \bruni
    \end{tikzpicture}
\end{center}
\hrule
\vspace{0.5cm}

Commande \textbf{b}loc + \textbf{r}etour \textbf{uni}taire (étendue) : 

\verb?\bruni[a][b][c][d][e]?
\begin{center}
    \begin{tikzpicture}
        \bruni[a][b][c][d][e]
    \end{tikzpicture}
\end{center}
\hrule
\vspace{0.5cm}

%%%%%%%%%%%%%%%%%%%%%%%%%%%%%%%%%%%%%%%%%%%%%%%%%%%%%%%%%%%%%%%%%%%%%%%%%%%%%%%
\subsection{Boucle de retour}
%%%%%%%%%%%%%%%%%%%%%%%%%%%%%%%%%%%%%%%%%%%%%%%%%%%%%%%%%%%%%%%%%%%%%%%%%%%%%%%
Commande \textbf{b}loc + \textbf{b}loc \textbf{r}etour : 

\verb?\bbr?
\begin{center}
    \begin{tikzpicture}
        \bbr
    \end{tikzpicture}
\end{center}
\hrule
\vspace{0.5cm}

Commande \textbf{b}loc + \textbf{b}loc \textbf{r}etour (étendue) : 

\verb?\bbr[a][b][c][d][e][f][g][h]?
\begin{center}
    \begin{tikzpicture}
        \bbr[a][b][c][d][e][f][g][h]
    \end{tikzpicture}
\end{center}
\hrule
\vspace{0.5cm}


%%%%%%%%%%%%%%%%%%%%%%%%%%%%%%%%%%%%%%%%%%%%%%%%%%%%%%%%%%%%%%%%%%%%%%%%%%%%%%%%
\subsection{Correcteur}
%%%%%%%%%%%%%%%%%%%%%%%%%%%%%%%%%%%%%%%%%%%%%%%%%%%%%%%%%%%%%%%%%%%%%%%%%%%%%%%%
Commande : \textbf{corr}ecteur + \textbf{b}loc + 
           \textbf{r}etour \textbf{uni}taire : 

\verb?\corrbruni?
\begin{center}
    \begin{tikzpicture}
        \corrbruni
    \end{tikzpicture}
\end{center}
\hrule
\vspace{0.5cm}

Commande : \textbf{corr}ecteur + \textbf{b}loc + 
           \textbf{r}etour \textbf{uni}taire (étendue) : 

\verb?\corrbruni[a][b][c][d][e][f][g][h]?
\begin{center}
    \begin{tikzpicture}
        \corrbruni[a][b][c][d][e][f][g][h]
    \end{tikzpicture}
\end{center}
\hrule
\vspace{0.5cm}


Commande :\textbf{corr}ecteur + \textbf{b}loc + \textbf{b}loc \textbf{r}etour: 
\verb?\corrbbr{}?
\begin{center}
    \begin{tikzpicture}
        \corrbbr{}
    \end{tikzpicture}
\end{center}
\hrule
\vspace{0.5cm}

Commande : \textbf{corr}ecteur + \textbf{b}loc + 
          \textbf{b}loc \textbf{r}etour (étendue) :

\verb?\corrbbr[a][b][c][d][e][f][g][h]{[i][j][k]}?
\begin{center}
    \begin{tikzpicture}
        \corrbbr[a][b][c][d][e][f][g][h]{[i][j][k]}
    \end{tikzpicture}
\end{center}
\hrule
\vspace{0.5cm}
\clearpage
%%%%%%%%%%%%%%%%%%%%%%%%%%%%%%%%%%%%%%%%%%%%%%%%%%%%%%%%%%%%%%%%%%%%%%%%%%%%%%%%
\subsection{Pertubation}
%%%%%%%%%%%%%%%%%%%%%%%%%%%%%%%%%%%%%%%%%%%%%%%%%%%%%%%%%%%%%%%%%%%%%%%%%%%%%%%%

Commande : \textbf{c}orrecteur + \textbf{p}erturbation + 
           \textbf{b}loc + 
           \textbf{r}etour \textbf{uni}taire: 

\verb?\cpbruni?
\begin{center}
    \begin{tikzpicture}
        \cpbruni
    \end{tikzpicture}
\end{center}
\hrule
\vspace{0.5cm}


Commande : \textbf{c}orrecteur + \textbf{p}erturbation + 
           \textbf{b}loc + 
           \textbf{r}etour \textbf{uni}taire (étendue) : 

\verb?\cpbruni[a][b][c][d][e][f][g][h][i]?
\begin{center}
    \begin{tikzpicture}
        \cpbruni[a][b][c][d][e][f][g][h][i]
    \end{tikzpicture}
\end{center}
\hrule
\vspace{0.5cm}

Commande : \textbf{c}orrecteur + \textbf{p}erturbation + 
           \textbf{b}loc + 
           \textbf{b}loc \textbf{r}etour : 

\verb?\cpbbr{}?
\begin{center}
    \begin{tikzpicture}
        \cpbbr{}
    \end{tikzpicture}
\end{center}
\hrule
\vspace{0.5cm}


Commande : \textbf{c}orrecteur + \textbf{p}erturbation + 
           \textbf{b}loc + 
           \textbf{b}loc \textbf{r}etour (étendue) : 

\verb?\cpbbr[a][b][c][d][e][f][g][h]{[i][j][k][l]}?
\begin{center}
    \begin{tikzpicture}
        \cpbbr[a][b][c][d][e][f][g][h]{[i][j][k][l]}
    \end{tikzpicture}
\end{center}
\hrule
\vspace{0.5cm}

%%%%%%%%%%%%%%%%%%%%%%%%%%%%%%%%%%%%%%%%%%%%%%%%%%%%%%%%%%%%%%%%%%%%%%%%%%%%%%%%
\subsection{Boucle retour unitaire (retour positif)}
%%%%%%%%%%%%%%%%%%%%%%%%%%%%%%%%%%%%%%%%%%%%%%%%%%%%%%%%%%%%%%%%%%%%%%%%%%%%%%%%
Commande \textbf{b}loc + \textbf{r}etour \textbf{uni}taire + \textbf{p}ositif : 

\verb?\brunip?
\begin{center}
    \begin{tikzpicture}
        \brunip
    \end{tikzpicture}
\end{center}
\hrule
\vspace{0.5cm}

Commande \textbf{b}loc + \textbf{r}etour \textbf{uni}taire (étendue) + \textbf{p}ositif : 

\verb?\brunip[a][b][c][d][e]?
\begin{center}
    \begin{tikzpicture}
        \brunip[a][b][c][d][e]
    \end{tikzpicture}
\end{center}
\hrule
\vspace{0.5cm}

%%%%%%%%%%%%%%%%%%%%%%%%%%%%%%%%%%%%%%%%%%%%%%%%%%%%%%%%%%%%%%%%%%%%%%%%%%%%%%%
\subsection{Boucle de retour (retour positif)}
%%%%%%%%%%%%%%%%%%%%%%%%%%%%%%%%%%%%%%%%%%%%%%%%%%%%%%%%%%%%%%%%%%%%%%%%%%%%%%%
Commande \textbf{b}loc + \textbf{b}loc \textbf{r}etour + \textbf{p}ositif : 

\verb?\bbrp?
\begin{center}
    \begin{tikzpicture}
        \bbrp
    \end{tikzpicture}
\end{center}
\hrule
\vspace{0.5cm}

Commande \textbf{b}loc + \textbf{b}loc \textbf{r}etour + \textbf{p}ositif (étendue) : 

\verb?\bbrp[a][b][c][d][e][f][g]?
\begin{center}
    \begin{tikzpicture}
        \bbrp[a][b][c][d][e][f][g][h]
    \end{tikzpicture}
\end{center}
\hrule
\vspace{0.5cm}


%%%%%%%%%%%%%%%%%%%%%%%%%%%%%%%%%%%%%%%%%%%%%%%%%%%%%%%%%%%%%%%%%%%%%%%%%%%%%%%%
\subsection{Correcteur (retour positif) }
%%%%%%%%%%%%%%%%%%%%%%%%%%%%%%%%%%%%%%%%%%%%%%%%%%%%%%%%%%%%%%%%%%%%%%%%%%%%%%%%
Commande : \textbf{corr}ecteur + \textbf{b}loc + 
           \textbf{r}etour \textbf{uni}taire + \textbf{p}ositif : 

\verb?\corrbrunip?
\begin{center}
    \begin{tikzpicture}
        \corrbrunip
    \end{tikzpicture}
\end{center}
\hrule
\vspace{0.5cm}

Commande : \textbf{corr}ecteur + \textbf{b}loc + 
           \textbf{r}etour \textbf{uni}taire + \textbf{p}ositif (étendue) : 

\verb?\corrbrunip[a][b][c][d][e][f][g][h]?
\begin{center}
    \begin{tikzpicture}
        \corrbrunip[a][b][c][d][e][f][g][h]
    \end{tikzpicture}
\end{center}
\hrule
\vspace{0.5cm}


Commande :\textbf{corr}ecteur + \textbf{b}loc + \textbf{b}loc \textbf{r}etour + \textbf{p}ositif: 
\verb?\corrbbrp{}?
\begin{center}
    \begin{tikzpicture}
        \corrbbrp{}
    \end{tikzpicture}
\end{center}
\hrule
\vspace{0.5cm}

Commande : \textbf{corr}ecteur + \textbf{b}loc + 
          \textbf{b}loc \textbf{r}etour + \textbf{p}ositif (étendue) :

\verb?\corrbbrp[a][b][c][d][e][f][g][h]{[i][j][k]}?
\begin{center}
    \begin{tikzpicture}
        \corrbbrp[a][b][c][d][e][f][g][h]{[i][j][k]}
    \end{tikzpicture}
\end{center}
\hrule
\vspace{0.5cm}
%%%%%%%%%%%%%%%%%%%%%%%%%%%%%%%%%%%%%%%%%%%%%%%%%%%%%%%%%%%%%%%%%%%%%%%%%%%%%%%%
\subsection{Pertubation (retour positif) }
%%%%%%%%%%%%%%%%%%%%%%%%%%%%%%%%%%%%%%%%%%%%%%%%%%%%%%%%%%%%%%%%%%%%%%%%%%%%%%%%

Commande : \textbf{c}orrecteur + \textbf{p}erturbation + 
           \textbf{b}loc + 
           \textbf{r}etour \textbf{uni}taire + \textbf{p}ositif : 

\verb?\cpbrunip?
\begin{center}
    \begin{tikzpicture}
        \cpbrunip
    \end{tikzpicture}
\end{center}
\hrule
\vspace{0.5cm}


Commande : \textbf{c}orrecteur + \textbf{p}erturbation + 
           \textbf{b}loc + 
           \textbf{r}etour \textbf{uni}taire + \textbf{p}ositif (étendue) : 

\verb?\cpbrunip[a][b][c][d][e][f][g][h][i]?
\begin{center}
    \begin{tikzpicture}
        \cpbrunip[a][b][c][d][e][f][g][h][i]
    \end{tikzpicture}
\end{center}
\hrule
\vspace{0.5cm}

Commande : \textbf{c}orrecteur + \textbf{p}erturbation + 
           \textbf{b}loc + 
           \textbf{b}loc \textbf{r}etour  + \textbf{p}ositif : 

\verb?\cpbbrp{}?
\begin{center}
    \begin{tikzpicture}
        \cpbbrp{}
    \end{tikzpicture}
\end{center}
\hrule
\vspace{0.5cm}


Commande : \textbf{c}orrecteur + \textbf{p}erturbation + 
           \textbf{b}loc + 
           \textbf{b}loc \textbf{r}etour  + \textbf{p}ositif (étendue) : 

\verb?\cpbbrp[a][b][c][d][e][f][g][h]{[i][j][k][l]}?
\begin{center}
    \begin{tikzpicture}
        \cpbbrp[a][b][c][d][e][f][g][h]{[i][j][k][l]}
    \end{tikzpicture}
\end{center}
\hrule
\vspace{0.5cm}
\clearpage
%%%%%%%%%%%%%%%%%%%%%%%%%%%%%%%%%%%%%%%%%%%%%%%%%%%%%%%%%%%%%%%%%%%%%%%%%%%%%%%%
\subsection{Options de styles de \texttt{schemabloc}}
%%%%%%%%%%%%%%%%%%%%%%%%%%%%%%%%%%%%%%%%%%%%%%%%%%%%%%%%%%%%%%%%%%%%%%%%%%%%%%%%
\sbStyleBloc{thick}
\sbStyleLien{thick}
\sbStyleSum{thick}

\begin{verbatim}
\sbStyleBloc{thick}
\sbStyleLien{thick}
\sbStyleSum{thick}
\end{verbatim}
\begin{center}
    \begin{tikzpicture}
        \cpbbr[a][b][c][d][e][f][g][h]{[i][j][k][l]}
    \end{tikzpicture}
\end{center}
\hrule
\vspace{0.5cm}

\sbStyleBloc{ultra thick,col3}
\sbStyleLien{ultra thick,col4}
\sbStyleSum{ultra thick,col1}

\begin{verbatim}
\sbStyleBloc{ultra thick,col3}
\sbStyleLien{ultra thick,col4}
\sbStyleSum{ultra thick,col1}
\end{verbatim}
\begin{center}
    \begin{tikzpicture}
        \cpbbr[a][b][c][d][e][f][g][h]{[i][j][k][l]}
    \end{tikzpicture}
\end{center}
\hrule
\vspace{0.5cm}

%%%%%%%%%%%%%%%%%%%%%%%%%%%%%%%%%%%%%%%%%%%%%%%%%%%%%%%%%%%%%%%%%%%%%%%%%%%%%%%%
\subsection{Schémas-bloc dans le domaine temporel}
%%%%%%%%%%%%%%%%%%%%%%%%%%%%%%%%%%%%%%%%%%%%%%%%%%%%%%%%%%%%%%%%%%%%%%%%%%%%%%%%

L'opérateur intégral $\mathcal{I}$ appliqué à un signal 
continu $x(t)$ permet de générer un signal $y(t)$ correspondant 
à l'intégral en tous points de $x(t)$.

\[
    y(t)=\mathcal{I} x(t)=\int_{-\infty}^t x(\tau)\dd{\tau}
\]
sous forme d'opérateur on écrira $Y=\mathcal{I}X$ de façon équivalente.

\sbStyleBlocDefaut
\sbStyleLienDefaut
\sbStyleSumDefaut
\sbStyleBlocT{ultra thick,col3}
\begin{center}
    \begin{tikzpicture}
        \sbEntree{E1}
        \sbInt[3]{I1}{E1}
        \sbRelier[$x(t)$]{E1}{I1}
        \sbSortie[3]{S1}{I1}
        \sbRelier[$y(t)$]{I1}{S1}
    \end{tikzpicture}
\end{center}
Nous representons le produit par un simple scalaire (c.-à-d. un gain) $s$ par 
le triangle de la façon suivante:

\begin{center}
\begin{tikzpicture}
    \sbEntree{E1}
    \sbBlocT[3]{T}{E1}
    \sbRelier[$x(t)$]{E1}{T}
    \sbSortie[3]{S1}{T}
    \sbRelier[$y(t)$]{T}{S1}
\end{tikzpicture}
\end{center}
\[
    y(t)=sx(t)
\]

L'équation différentielle suivante :
\[
    \dot{y}(t)=s\big(x(t)+y(t)\big)
\]
peut alors être représentée par le schéma-bloc dans le domaine temporel suivant :

\begin{center}
\begin{tikzpicture}
    \sbEntree{E1}
    \sbCompSum[5.0]{comp}{E1}{}{+}{+}{}
    \sbRelier[$x(t)$]{E1}{comp}
    \sbBlocT[3]{T}{comp}
    \sbRelier[]{comp}{T}
    \sbInt[3]{B1}{T}
    \sbRelier[$\dot{y}(t)$]{T}{B1}
    \sbSortie[6]{S1}{B1}
    \sbRelier[$y(t)$]{B1}{S1}
    \sbRenvoi[5]{B1-S1}{comp}{}
\end{tikzpicture}
\end{center}

sous forme d'opérateur, on réduira ce schéma-bloc par la relation suivante :
\[
    Y=s\mathcal{I}\big(X+Y\big)
\]
Le rapport de la sortie sur l'entrée
\[
    \dfrac{Y}{X}=\dfrac{s\mathcal{I}}{1-s\mathcal{I}}
\]
L'équation différentielle (2) :
\[
    \dot{y}(t)=x+sy(t)
\]

\begin{center}
\begin{tikzpicture}
    \sbEntree{E1}
    \sbCompSum[5.0]{comp}{E1}{}{+}{+}{}
    \sbRelier[$x(t)$]{E1}{comp}
    \sbInt[3]{B1}{comp}
    \sbRelier[$\dot{y}(t)$]{comp}{B1}
    \sbSortie[6]{S1}{B1}
    \sbRelier[$y(t)$]{B1}{S1}
    \sbDecaleNoeudy{B1}{R}
    \sbBlocTr[-1.5]{R1}{R}
    \sbRelieryx{B1-S1}{R1}
    \sbRelierxy[]{R1}{comp}
\end{tikzpicture}
\end{center}

Sous forme d'opérateur
\[
    Y=\mathcal{I}\big(X+sY\big)
\]
\[
    \dfrac{Y}{X}=\dfrac{\mathcal{I}}{1-s\mathcal{I}}
\]
L'équation différentielle (3) :
\[
    \dot{y}(t)=\dot{x}(t)+sy(t)
\]
\begin{center}
\begin{tikzpicture}
    \sbEntree{E1}
    \sbCompSum[5.0]{comp}{E1}{}{+}{+}{}
    \sbRelier[$x(t)$]{E1}{comp}
    \sbSortie[12]{S1}{comp}
    \node [left of =S1, node distance=4em] (ref) {};
    \sbRelierNoDraw[]{ref}{S1}
    \sbRelier[$y(t)$]{comp}{S1}
    \sbDecaleNoeudy{S1}{R}
    \sbIntr[4]{B1}{R}
    \sbRelieryx{ref-S1}{B1}
    \sbBlocTr[2.5]{R1}{B1}
    \sbRelier[]{B1}{R1}
    \sbRelierxy[]{R1}{comp}
\end{tikzpicture}
\end{center}
Sous forme d'opérateur
\[
    Y=X+s\mathcal{I}Y
\]
\[
    \dfrac{Y}{X}=\dfrac{1}{1-s\mathcal{I}}
\]
Il est possible de développer $\dfrac{1}{1-s\mathcal{I}}$ sous forme de série entière :
\[
    \dfrac{1}{1-s\mathcal{I}}=\sum_{n=0}^{\infty}(s\mathcal{I})^n=(1+s\mathcal{I}+s^2\mathcal{I}^2+s^3\mathcal{I}^3\ldots)
\]
L'application de l'opérateur $\mathcal{I}^n$ revient à appliquer $\mathcal{I}$ $n$ fois.

Prenons l'exemple de l'équation différentielle :
\[
    \dot{y}(t)=x+sy(t)
\]
en appliquant la transformée de Laplace on obtiendrait :
\[
    Y(p)=\dfrac{1}{p-s}X(p)
\]

Pour une entrée $x(t)=\delta(t)$, la sortie devient dans le domaine de Laplace $Y(p)=\dfrac{1}{p-s}$, par la transformée de Laplace
inverse celà devient $y(t)=e^{-st}u(t)$.




\end{document}
