\documentclass[a4paper,9pt]{article} 
\usepackage[utf8]{inputenc}          
\usepackage[T1]{fontenc}             
\usepackage[francais]{babel}         
\usepackage{lmodern}                 

\usepackage{geometry}                                                                                                         
\geometry{hmargin=2.5cm,vmargin=2.6cm}

%\usepackage{schemabloc}
\usepackage{schemablocfmv}
\usepackage{mathfmv}
\usepackage{amsmath}
\usepackage{titling}
\usepackage{tikz,pgf}
\usepackage{verbatim}
\usetikzlibrary{calc}

\newcommand{\subtitle}[1]{%                          
\posttitle{%                                   
\par\end{center}%
\begin{center}\large#1\end{center}%
\vskip0.5em}%              
}                          
\newcommand{\mparagraph}[1]{\paragraph{#1}\mbox{}\\} 
\title{Extension de \texttt{schemabloc} avec PGF/Tikz}
\subtitle{schemablocfmv version 0.1}
\author{F. M. Vasconcelos}
\date{}

\begin{document}
\maketitle
\begin{abstract}
Ce paquet définit des macros à partir du paquet \texttt{schemabloc}
de R. Papanicola.
\end{abstract}

%%%%%%%%%%%%%%%%%%%%%%%%%%%%%%%%%%%%%%%%%%%%%%%%%%%%%%%%%%%%%%%%%%%%%%%%%%%%%%%%%     
\section{Paquets nécessaires}                                                         
%%%%%%%%%%%%%%%%%%%%%%%%%%%%%%%%%%%%%%%%%%%%%%%%%%%%%%%%%%%%%%%%%%%%%%%%%%%%%%%%%     
\begin{verbatim}                                                                      
\usepackage{schemabloc}
\usepackage{schemablocfmv}
\end{verbatim} 
%%%%%%%%%%%%%%%%%%%%%%%%%%%%%%%%%%%%%%%%%%%%%%%%%%%%%%%%%%%%%%%%%%%%%%%%%%%%%%%%%     
\section{Utilisation des macros}                                                         
%%%%%%%%%%%%%%%%%%%%%%%%%%%%%%%%%%%%%%%%%%%%%%%%%%%%%%%%%%%%%%%%%%%%%%%%%%%%%%%%%     

\subsection{Bloc simple}
Commande \textbf{b}loc \textbf{s}imple: \verb?\bs? 
\begin{center}
    \begin{tikzpicture}
        \bs
    \end{tikzpicture}
\end{center}
\hrule
\vspace{0.5cm}

Commande \textbf{b}loc \textbf{s}imple (étendue) : \verb?\bs[a][b][c][d]?
\begin{center}
    \begin{tikzpicture}
        \bs[a][b][c][d]
    \end{tikzpicture}
\end{center}
\hrule
\vspace{0.5cm}

\subsection{Comparateur}
Commande \textbf{comp}arateur + \textbf{b}loc \textbf{s}imple : \verb?\compbs?
\begin{center}
    \begin{tikzpicture}
        \compbs
    \end{tikzpicture}
\end{center}
\hrule
\vspace{0.5cm}

Commande \textbf{comp}arateur + \textbf{b}loc \textbf{s}imple (étendue) : \verb?\compbs[a][b][c][d][e][f]?
\begin{center}
    \begin{tikzpicture}
        \compbs[a][b][c][d][e][f]
    \end{tikzpicture}
\end{center}
\hrule
\vspace{0.5cm}

\clearpage

\subsection{Boucle retour unitaire}
Commande \textbf{b}loc + \textbf{r}etour \textbf{uni}taire: \verb?\bruni?
\begin{center}
    \begin{tikzpicture}
        \bruni
    \end{tikzpicture}
\end{center}
\hrule
\vspace{0.5cm}

Commande \textbf{b}loc + \textbf{r}etour \textbf{uni}taire (étendue) : \verb?\bruni[a][b][c][d][e]?
\begin{center}
    \begin{tikzpicture}
        \bruni[a][b][c][d][e]
    \end{tikzpicture}
\end{center}
\hrule
\vspace{0.5cm}

\subsection{Boucle de retour}
Commande \textbf{b}loc + \textbf{b}loc \textbf{r}etour : \verb?\bbr?
\begin{center}
    \begin{tikzpicture}
        \bbr
    \end{tikzpicture}
\end{center}
\hrule
\vspace{0.5cm}

Commande \textbf{b}loc + \textbf{b}loc \textbf{r}etour (étendue) : \verb?\bbr[a][b][c][d][e][f][g]?
\begin{center}
    \begin{tikzpicture}
        \bbr[a][b][c][d][e][f][g][h]
    \end{tikzpicture}
\end{center}
\hrule
\vspace{0.5cm}


\subsection{Correcteur}
Commande : \textbf{corr}ecteur + \textbf{b}loc + \textbf{r}etour \textbf{uni}taire : \verb?\corrbruni?
\begin{center}
    \begin{tikzpicture}
        \corrbruni
    \end{tikzpicture}
\end{center}
\hrule
\vspace{0.5cm}

Commande : \textbf{corr}ecteur + \textbf{b}loc + \textbf{r}etour \textbf{uni}taire (étendue) : \verb?\corrbruni[a][b][c][d][e][f][g][h]?
\begin{center}
    \begin{tikzpicture}
        \corrbruni[a][b][c][d][e][f][g][h]
    \end{tikzpicture}
\end{center}
\hrule
\vspace{0.5cm}


Commande : \textbf{corr}ecteur + \textbf{b}loc + \textbf{b}loc \textbf{r}etour : \verb?\corrbbr{}?
\begin{center}
    \begin{tikzpicture}
        \corrbbr{}
    \end{tikzpicture}
\end{center}
\hrule
\vspace{0.5cm}

Commande : \textbf{corr}ecteur + \textbf{b}loc + \textbf{r}etour \textbf{uni}taire (étendue) : \verb?\corrbbr[a][b][c][d][e][f][g][h]{[i][j][k]}?
\begin{center}
    \begin{tikzpicture}
        \corrbbr[a][b][c][d][e][f][g][h]{[i][j][k]}
    \end{tikzpicture}
\end{center}
\hrule
\vspace{0.5cm}




\end{document}
