\documentclass[a4paper,10pt]{article} 
\usepackage[utf8]{inputenc}          
\usepackage[T1]{fontenc}             
\usepackage[francais]{babel}         
\usepackage{lmodern}                 

\usepackage{geometry}                                                                                                         
\geometry{hmargin=2.5cm,vmargin=2.6cm}

\usepackage{tikz,pgf,pgfplots}
\pgfplotsset{compat=newest}
\usepackage{tikzfmv}
\usepackage{schemablocfmv}
\usepackage{mathfmv}
\usepackage{amsmath}
\usepackage{nccmath}
\usepackage{titling}
\usepackage{verbatim}
\usetikzlibrary{calc}
\usetikzlibrary{shapes.geometric}

\newcommand{\subtitle}[1]{%                          
\posttitle{%                                   
\par\end{center}%
\begin{center}\large#1\end{center}%
\vskip0.5em}%              
}                          
\newcommand{\mparagraph}[1]{\paragraph{#1}\mbox{}\\} 
\title{Extension de \texttt{schemabloc} avec PGF/Tikz}
\subtitle{schemablocfmv version 0.2}
\author{F. M. Vasconcelos}
\date{}

\begin{document}
\maketitle
\begin{abstract}
Ce paquet définit des macros à partir du paquet \texttt{schemabloc}
de R. Papanicola.
\end{abstract}

%%%%%%%%%%%%%%%%%%%%%%%%%%%%%%%%%%%%%%%%%%%%%%%%%%%%%%%%%%%%%%%%%%%%%%%%%%%%%%%%
\section{Paquets nécessaires}
%%%%%%%%%%%%%%%%%%%%%%%%%%%%%%%%%%%%%%%%%%%%%%%%%%%%%%%%%%%%%%%%%%%%%%%%%%%%%%%%

\begin{verbatim}
\usepackage{tikz}
\usepackage{schemabloc}
\usepackage{schemablocfmv}
\end{verbatim} 

%%%%%%%%%%%%%%%%%%%%%%%%%%%%%%%%%%%%%%%%%%%%%%%%%%%%%%%%%%%%%%%%%%%%%%%%%%%%%%%%
\section{Utilisation des macros}
%%%%%%%%%%%%%%%%%%%%%%%%%%%%%%%%%%%%%%%%%%%%%%%%%%%%%%%%%%%%%%%%%%%%%%%%%%%%%%%%

%%%%%%%%%%%%%%%%%%%%%%%%%%%%%%%%%%%%%%%%%%%%%%%%%%%%%%%%%%%%%%%%%%%%%%%%%%%%%%%
\subsection{Bloc simple}
%%%%%%%%%%%%%%%%%%%%%%%%%%%%%%%%%%%%%%%%%%%%%%%%%%%%%%%%%%%%%%%%%%%%%%%%%%%%%%%
Commande \textbf{b}loc \textbf{s}imple: 

\verb?\bs?
\begin{center}
    \begin{tikzpicture}
        \bs
    \end{tikzpicture}
\end{center}
\hrule
\vspace{0.5cm}

Commande \textbf{b}loc \textbf{s}imple (étendue) : 

\verb?\bs[a][b][c][d]?
\begin{center}
    \begin{tikzpicture}
        \bs[a][b][c][d]
    \end{tikzpicture}
\end{center}
\hrule
\vspace{0.5cm}
\clearpage
%%%%%%%%%%%%%%%%%%%%%%%%%%%%%%%%%%%%%%%%%%%%%%%%%%%%%%%%%%%%%%%%%%%%%%%%%%%%%%%
\subsection{Comparateur}
%%%%%%%%%%%%%%%%%%%%%%%%%%%%%%%%%%%%%%%%%%%%%%%%%%%%%%%%%%%%%%%%%%%%%%%%%%%%%%%
Commande \textbf{comp}arateur + \textbf{b}loc \textbf{s}imple : 

\verb?\compbs?
\begin{center}
    \begin{tikzpicture}
        \compbs
    \end{tikzpicture}
\end{center}
\hrule
\vspace{0.5cm}

Commande \textbf{comp}arateur + \textbf{b}loc \textbf{s}imple (étendue) : 

\verb?\compbs[a][b][c][d][e][f]?
\begin{center}
    \begin{tikzpicture}
        \compbs[a][b][c][d][e][f]
    \end{tikzpicture}
\end{center}
\hrule
\vspace{0.5cm}
%%%%%%%%%%%%%%%%%%%%%%%%%%%%%%%%%%%%%%%%%%%%%%%%%%%%%%%%%%%%%%%%%%%%%%%%%%%%%%%
\subsection{Sommateur}
%%%%%%%%%%%%%%%%%%%%%%%%%%%%%%%%%%%%%%%%%%%%%%%%%%%%%%%%%%%%%%%%%%%%%%%%%%%%%%%
Commande \textbf{som}ateur + \textbf{b}loc \textbf{s}imple : 

\verb?\sombs?
\begin{center}
    \begin{tikzpicture}
        \sombs
    \end{tikzpicture}
\end{center}
\hrule
\vspace{0.5cm}

Commande \textbf{som}arateur + \textbf{b}loc \textbf{s}imple (étendue) : 

\verb?\sombs[a][b][c][d][e][f]?
\begin{center}
    \begin{tikzpicture}
        \sombs[a][b][c][d][e][f]
    \end{tikzpicture}
\end{center}
\hrule
\vspace{0.5cm}
\clearpage
%%%%%%%%%%%%%%%%%%%%%%%%%%%%%%%%%%%%%%%%%%%%%%%%%%%%%%%%%%%%%%%%%%%%%%%%%%%%%%%%
\subsection{Boucle retour unitaire}
%%%%%%%%%%%%%%%%%%%%%%%%%%%%%%%%%%%%%%%%%%%%%%%%%%%%%%%%%%%%%%%%%%%%%%%%%%%%%%%%
Commande \textbf{b}loc + \textbf{r}etour \textbf{uni}taire: 

\verb?\bruni?
\begin{center}
    \begin{tikzpicture}
        \bruni
    \end{tikzpicture}
\end{center}
\hrule
\vspace{0.5cm}

Commande \textbf{b}loc + \textbf{r}etour \textbf{uni}taire (étendue) : 

\verb?\bruni[a][b][c][d][e]?
\begin{center}
    \begin{tikzpicture}
        \bruni[a][b][c][d][e]
    \end{tikzpicture}
\end{center}
\hrule
\vspace{0.5cm}

%%%%%%%%%%%%%%%%%%%%%%%%%%%%%%%%%%%%%%%%%%%%%%%%%%%%%%%%%%%%%%%%%%%%%%%%%%%%%%%
\subsection{Boucle de retour}
%%%%%%%%%%%%%%%%%%%%%%%%%%%%%%%%%%%%%%%%%%%%%%%%%%%%%%%%%%%%%%%%%%%%%%%%%%%%%%%
Commande \textbf{b}loc + \textbf{b}loc \textbf{r}etour : 

\verb?\bbr?
\begin{center}
    \begin{tikzpicture}
        \bbr
    \end{tikzpicture}
\end{center}
\hrule
\vspace{0.5cm}

Commande \textbf{b}loc + \textbf{b}loc \textbf{r}etour (étendue) : 

\verb?\bbr[a][b][c][d][e][f][g][h]?
\begin{center}
    \begin{tikzpicture}
        \bbr[a][b][c][d][e][f][g][h]
    \end{tikzpicture}
\end{center}
\hrule
\vspace{0.5cm}


%%%%%%%%%%%%%%%%%%%%%%%%%%%%%%%%%%%%%%%%%%%%%%%%%%%%%%%%%%%%%%%%%%%%%%%%%%%%%%%%
\subsection{Correcteur}
%%%%%%%%%%%%%%%%%%%%%%%%%%%%%%%%%%%%%%%%%%%%%%%%%%%%%%%%%%%%%%%%%%%%%%%%%%%%%%%%
Commande : \textbf{corr}ecteur + \textbf{b}loc + 
           \textbf{r}etour \textbf{uni}taire : 

\verb?\corrbruni?
\begin{center}
    \begin{tikzpicture}
        \corrbruni
    \end{tikzpicture}
\end{center}
\hrule
\vspace{0.5cm}

Commande : \textbf{corr}ecteur + \textbf{b}loc + 
           \textbf{r}etour \textbf{uni}taire (étendue) : 

\verb?\corrbruni[a][b][c][d][e][f][g][h]?
\begin{center}
    \begin{tikzpicture}
        \corrbruni[a][b][c][d][e][f][g][h]
    \end{tikzpicture}
\end{center}
\hrule
\vspace{0.5cm}


Commande :\textbf{corr}ecteur + \textbf{b}loc + \textbf{b}loc \textbf{r}etour: 
\verb?\corrbbr{}?
\begin{center}
    \begin{tikzpicture}
        \corrbbr{}
    \end{tikzpicture}
\end{center}
\hrule
\vspace{0.5cm}

Commande : \textbf{corr}ecteur + \textbf{b}loc + 
          \textbf{b}loc \textbf{r}etour (étendue) :

\verb?\corrbbr[a][b][c][d][e][f][g][h]{[i][j][k]}?
\begin{center}
    \begin{tikzpicture}
        \corrbbr[a][b][c][d][e][f][g][h]{[i][j][k]}
    \end{tikzpicture}
\end{center}
\hrule
\vspace{0.5cm}
\clearpage
%%%%%%%%%%%%%%%%%%%%%%%%%%%%%%%%%%%%%%%%%%%%%%%%%%%%%%%%%%%%%%%%%%%%%%%%%%%%%%%%
\subsection{Pertubation}
%%%%%%%%%%%%%%%%%%%%%%%%%%%%%%%%%%%%%%%%%%%%%%%%%%%%%%%%%%%%%%%%%%%%%%%%%%%%%%%%

Commande : \textbf{c}orrecteur + \textbf{p}erturbation + 
           \textbf{b}loc + 
           \textbf{r}etour \textbf{uni}taire: 

\verb?\cpbruni?
\begin{center}
    \begin{tikzpicture}
        \cpbruni
    \end{tikzpicture}
\end{center}
\hrule
\vspace{0.5cm}


Commande : \textbf{c}orrecteur + \textbf{p}erturbation + 
           \textbf{b}loc + 
           \textbf{r}etour \textbf{uni}taire (étendue) : 

\verb?\cpbruni[a][b][c][d][e][f][g][h][i]?
\begin{center}
    \begin{tikzpicture}
        \cpbruni[a][b][c][d][e][f][g][h][i]
    \end{tikzpicture}
\end{center}
\hrule
\vspace{0.5cm}

Commande : \textbf{c}orrecteur + \textbf{p}erturbation + 
           \textbf{b}loc + 
           \textbf{b}loc \textbf{r}etour : 

\verb?\cpbbr{}?
\begin{center}
    \begin{tikzpicture}
        \cpbbr{}
    \end{tikzpicture}
\end{center}
\hrule
\vspace{0.5cm}


Commande : \textbf{c}orrecteur + \textbf{p}erturbation + 
           \textbf{b}loc + 
           \textbf{b}loc \textbf{r}etour (étendue) : 

\verb?\cpbbr[a][b][c][d][e][f][g][h]{[i][j][k][l]}?
\begin{center}
    \begin{tikzpicture}
        \cpbbr[a][b][c][d][e][f][g][h]{[i][j][k][l]}
    \end{tikzpicture}
\end{center}
\hrule
\vspace{0.5cm}

%%%%%%%%%%%%%%%%%%%%%%%%%%%%%%%%%%%%%%%%%%%%%%%%%%%%%%%%%%%%%%%%%%%%%%%%%%%%%%%%
\subsection{Boucle retour unitaire (retour positif)}
%%%%%%%%%%%%%%%%%%%%%%%%%%%%%%%%%%%%%%%%%%%%%%%%%%%%%%%%%%%%%%%%%%%%%%%%%%%%%%%%
Commande \textbf{b}loc + \textbf{r}etour \textbf{uni}taire + \textbf{p}ositif : 

\verb?\brunip?
\begin{center}
    \begin{tikzpicture}
        \brunip
    \end{tikzpicture}
\end{center}
\hrule
\vspace{0.5cm}

Commande \textbf{b}loc + \textbf{r}etour \textbf{uni}taire (étendue) + \textbf{p}ositif : 

\verb?\brunip[a][b][c][d][e]?
\begin{center}
    \begin{tikzpicture}
        \brunip[a][b][c][d][e]
    \end{tikzpicture}
\end{center}
\hrule
\vspace{0.5cm}

%%%%%%%%%%%%%%%%%%%%%%%%%%%%%%%%%%%%%%%%%%%%%%%%%%%%%%%%%%%%%%%%%%%%%%%%%%%%%%%
\subsection{Boucle de retour (retour positif)}
%%%%%%%%%%%%%%%%%%%%%%%%%%%%%%%%%%%%%%%%%%%%%%%%%%%%%%%%%%%%%%%%%%%%%%%%%%%%%%%
Commande \textbf{b}loc + \textbf{b}loc \textbf{r}etour + \textbf{p}ositif : 

\verb?\bbrp?
\begin{center}
    \begin{tikzpicture}
        \bbrp
    \end{tikzpicture}
\end{center}
\hrule
\vspace{0.5cm}

Commande \textbf{b}loc + \textbf{b}loc \textbf{r}etour + \textbf{p}ositif (étendue) : 

\verb?\bbrp[a][b][c][d][e][f][g]?
\begin{center}
    \begin{tikzpicture}
        \bbrp[a][b][c][d][e][f][g][h]
    \end{tikzpicture}
\end{center}
\hrule
\vspace{0.5cm}


%%%%%%%%%%%%%%%%%%%%%%%%%%%%%%%%%%%%%%%%%%%%%%%%%%%%%%%%%%%%%%%%%%%%%%%%%%%%%%%%
\subsection{Correcteur (retour positif) }
%%%%%%%%%%%%%%%%%%%%%%%%%%%%%%%%%%%%%%%%%%%%%%%%%%%%%%%%%%%%%%%%%%%%%%%%%%%%%%%%
Commande : \textbf{corr}ecteur + \textbf{b}loc + 
           \textbf{r}etour \textbf{uni}taire + \textbf{p}ositif : 

\verb?\corrbrunip?
\begin{center}
    \begin{tikzpicture}
        \corrbrunip
    \end{tikzpicture}
\end{center}
\hrule
\vspace{0.5cm}

Commande : \textbf{corr}ecteur + \textbf{b}loc + 
           \textbf{r}etour \textbf{uni}taire + \textbf{p}ositif (étendue) : 

\verb?\corrbrunip[a][b][c][d][e][f][g][h]?
\begin{center}
    \begin{tikzpicture}
        \corrbrunip[a][b][c][d][e][f][g][h]
    \end{tikzpicture}
\end{center}
\hrule
\vspace{0.5cm}


Commande :\textbf{corr}ecteur + \textbf{b}loc + \textbf{b}loc \textbf{r}etour + \textbf{p}ositif: 
\verb?\corrbbrp{}?
\begin{center}
    \begin{tikzpicture}
        \corrbbrp{}
    \end{tikzpicture}
\end{center}
\hrule
\vspace{0.5cm}

Commande : \textbf{corr}ecteur + \textbf{b}loc + 
          \textbf{b}loc \textbf{r}etour + \textbf{p}ositif (étendue) :

\verb?\corrbbrp[a][b][c][d][e][f][g][h]{[i][j][k]}?
\begin{center}
    \begin{tikzpicture}
        \corrbbrp[a][b][c][d][e][f][g][h]{[i][j][k]}
    \end{tikzpicture}
\end{center}
\hrule
\vspace{0.5cm}
%%%%%%%%%%%%%%%%%%%%%%%%%%%%%%%%%%%%%%%%%%%%%%%%%%%%%%%%%%%%%%%%%%%%%%%%%%%%%%%%
\subsection{Pertubation (retour positif) }
%%%%%%%%%%%%%%%%%%%%%%%%%%%%%%%%%%%%%%%%%%%%%%%%%%%%%%%%%%%%%%%%%%%%%%%%%%%%%%%%

Commande : \textbf{c}orrecteur + \textbf{p}erturbation + 
           \textbf{b}loc + 
           \textbf{r}etour \textbf{uni}taire + \textbf{p}ositif : 

\verb?\cpbrunip?
\begin{center}
    \begin{tikzpicture}
        \cpbrunip
    \end{tikzpicture}
\end{center}
\hrule
\vspace{0.5cm}


Commande : \textbf{c}orrecteur + \textbf{p}erturbation + 
           \textbf{b}loc + 
           \textbf{r}etour \textbf{uni}taire + \textbf{p}ositif (étendue) : 

\verb?\cpbrunip[a][b][c][d][e][f][g][h][i]?
\begin{center}
    \begin{tikzpicture}
        \cpbrunip[a][b][c][d][e][f][g][h][i]
    \end{tikzpicture}
\end{center}
\hrule
\vspace{0.5cm}

Commande : \textbf{c}orrecteur + \textbf{p}erturbation + 
           \textbf{b}loc + 
           \textbf{b}loc \textbf{r}etour  + \textbf{p}ositif : 

\verb?\cpbbrp{}?
\begin{center}
    \begin{tikzpicture}
        \cpbbrp{}
    \end{tikzpicture}
\end{center}
\hrule
\vspace{0.5cm}


Commande : \textbf{c}orrecteur + \textbf{p}erturbation + 
           \textbf{b}loc + 
           \textbf{b}loc \textbf{r}etour  + \textbf{p}ositif (étendue) : 

\verb?\cpbbrp[a][b][c][d][e][f][g][h]{[i][j][k][l]}?
\begin{center}
    \begin{tikzpicture}
        \cpbbrp[a][b][c][d][e][f][g][h]{[i][j][k][l]}
    \end{tikzpicture}
\end{center}
\hrule
\vspace{0.5cm}
\clearpage
%%%%%%%%%%%%%%%%%%%%%%%%%%%%%%%%%%%%%%%%%%%%%%%%%%%%%%%%%%%%%%%%%%%%%%%%%%%%%%%%
\subsection{Options de styles de \texttt{schemabloc}}
%%%%%%%%%%%%%%%%%%%%%%%%%%%%%%%%%%%%%%%%%%%%%%%%%%%%%%%%%%%%%%%%%%%%%%%%%%%%%%%%
\sbStyleBloc{thick}
\sbStyleLien{thick}
\sbStyleSum{thick}

\begin{verbatim}
\sbStyleBloc{thick}
\sbStyleLien{thick}
\sbStyleSum{thick}
\end{verbatim}
\begin{center}
    \begin{tikzpicture}
        \cpbbr[a][b][c][d][e][f][g][h]{[i][j][k][l]}
    \end{tikzpicture}
\end{center}
\hrule
\vspace{0.5cm}

\sbStyleBloc{ultra thick,col3}
\sbStyleLien{ultra thick,col4}
\sbStyleSum{ultra thick,col1}

\begin{verbatim}
\sbStyleBloc{ultra thick,col3}
\sbStyleLien{ultra thick,col4}
\sbStyleSum{ultra thick,col1}
\end{verbatim}
\begin{center}
    \begin{tikzpicture}
        \cpbbr[a][b][c][d][e][f][g][h]{[i][j][k][l]}
    \end{tikzpicture}
\end{center}
\hrule
\vspace{0.5cm}

\section*{Schémas-bloc dans le domaine temporel}

L'opérateur intégral $\mathcal{I}$ appliqué à un signal 
continu $x(t)$ permet de générer un signal $y(t)$ correspondant 
à l'intégral en tous points de $x(t)$.

\[
    y(t)=\mathcal{I} x(t)=\int_{-\infty}^t x(\tau)\dd{\tau}
\]

\sbStyleBlocDefaut
\sbStyleLienDefaut
\sbStyleSumDefaut
\begin{center}

\end{center}


\begin{center}
\begin{tikzpicture}
    \sbEntree{E1}
    \sbCompSum[5.0]{comp}{E1}{}{+}{+}{}
    \sbRelier[$x(t)$]{E1}{comp}
    \sbInt[3]{B1}{comp}
    \sbRelier[$\dfrac{\dot{y}(t)}{p}$]{comp}{B1}
    \sbBlocT[3]{T}{B1}
    \sbRelier[$\dfrac{y(t)}{p}$]{B1}{T}
    \sbSortie[6]{S1}{T}
    \sbRelier[$y(t)$]{T}{S1}
    \sbRenvoi[5]{T-S1}{comp}{}
\end{tikzpicture}
\end{center}

\end{document}
