\documentclass[a4paper,9pt]{article}
\usepackage[utf8]{inputenc}
\usepackage[T1]{fontenc}
\usepackage[francais]{babel}
\usepackage{lmodern}

\usepackage{titling}
\usepackage{tikz}
\usepackage{xargs}
\usepackage{etoolbox}% http://ctan.org/pkg/etoolbox
\usepackage{nntikz}
\usepackage{verbatim}

\newcommand{\subtitle}[1]{%
      \posttitle{%
              \par\end{center}
                  \begin{center}\large#1\end{center}
                          \vskip0.5em}%
                          }
\newcommand{\mparagraph}[1]{\paragraph{#1}\mbox{}\\}

\title{Réseau de neurones artificiels avec PGF/Tikz}
\subtitle{nntikz version 1.0}
\author{F. M. Vasconcelos}
\date{}

\begin{document}
\maketitle
\begin{abstract}
Macros tikz pour tracer des graphes de réseau de neurones artificiels. 
\end{abstract}


%%%%%%%%%%%%%%%%%%%%%%%%%%%%%%%%%%%%%%%%%%%%%%%%%%%%%%%%%%%%%%%%%%%%%%%%%%%%%%
\section{Paquets nécessaires}
%%%%%%%%%%%%%%%%%%%%%%%%%%%%%%%%%%%%%%%%%%%%%%%%%%%%%%%%%%%%%%%%%%%%%%%%%%%%%%

%%%%%%%%%%%%%%%%%%%%%%%%%%%%%%%%%%%%%%%%%%%%%%%%%%%%%%%%%%%%%%%%%%%%%%%%%%%%%%
\section{Utilisation des macros}
%%%%%%%%%%%%%%%%%%%%%%%%%%%%%%%%%%%%%%%%%%%%%%%%%%%%%%%%%%%%%%%%%%%%%%%%%%%%%%
% ---------------------------------------------------------------------
\begin{center}
\begin{tikzpicture}
\nninput{nI}[1]
\nnoutput{nO}[1][1]
\nnfull{nI}{nO}
\end{tikzpicture}
\end{center}
\begin{verbatim}
\begin{tikzpicture}
\nninput{nI}[1]
\nnoutput{nO}[1][1]
\nnfull{nI}{nO}
\end{tikzpicture}
\end{verbatim}
\vspace{1cm}
% ---------------------------------------------------------------------
\begin{center}
\begin{tikzpicture}
\nninput{nI}[3]
\nnoutput{nO}[4][1]
\nnfull{nI}{nO}
\end{tikzpicture}
\end{center}
\begin{verbatim}
\begin{tikzpicture}
\nninput{nI}[3]
\nnoutput{nO}[4][1]
\nnfull{nI}{nO}
\end{tikzpicture}
\end{verbatim}
\vspace{1cm}
% ---------------------------------------------------------------------


\begin{center}
\begin{tikzpicture}
\nninput{nI}[2]
\nnhidden{nH1}[2][1]
\nnoutput{nO}[1][2]
\nnfull{nI}{nH1}
\nnfull{nH1}{nO}
\end{tikzpicture}
\end{center}
\begin{verbatim}
\begin{tikzpicture}
\nninput{nI}[2]
\nnhidden{nH1}[2][1]
\nnoutput{nO}[1][2]
\nnfull{nI}{nH1}
\nnfull{nH1}{nO}
\end{tikzpicture}
\end{verbatim}
\vspace{1cm}
% ---------------------------------------------------------------------
\begin{center}
\begin{tikzpicture}
\gdef\yoff{1.5}
\gdef\xoff{1.5}
\neuron{minimum size=20pt}
\nninput{nI}[2][label]
\nnhidden{nH1}[2][1][label]
\nnoutput{nO}[1][2][label]
\nnfull{nI}{nH1}
\nnfull{nH1}{nO}
\end{tikzpicture}
\end{center}
\begin{verbatim}
\begin{tikzpicture}
\gdef\yoff{1.5}
\gdef\xoff{1.5}
\neuron{minimum size=20pt}
\nninput{nI}[2][label]
\nnhidden{nH1}[2][1][label]
\nnoutput{nO}[1][2][label]
\nnfull{nI}{nH1}
\nnfull{nH1}{nO}
\end{tikzpicture}
\end{verbatim}
\vspace{1cm}
% ---------------------------------------------------------------------
\vspace{1cm}
\begin{center}
\begin{tikzpicture}
\gdef\yoff{1}
\gdef\xoff{1}
\nninput{nI}[2]
\nnhidden{nH1}[2][1]
\nnoutput{nO}[1][2]
\nnfull{nI}{nH1}
\nnfull{nH1}{nO}
\end{tikzpicture}
\end{center}
\begin{verbatim}
\begin{tikzpicture}
\gdef\yoff{1}
\gdef\xoff{1}
\nninput{nI}[2]
\nnhidden{nH1}[2][1]
\nnoutput{nO}[1][2]
\nnfull{nI}{nH1}
\nnfull{nH1}{nO}
\end{tikzpicture}
\end{verbatim}
\vspace{1cm}
% ---------------------------------------------------------------------
\begin{center}
\begin{tikzpicture}
\gdef\yoff{1.5}
\gdef\xoff{1.5}
\neuron{minimum size=10pt,fill=orange!50}
\nninput{nI}[2][label]
\neuron{minimum size=15pt,fill=yellow!50}
\nnhidden{nH1}[3][1]
\neuron{minimum size=25pt,fill=gray!50}
\nnoutput{nO}[1][2]
\nnfull{nI}{nH1}
\nnfull{nH1}{nO}
\end{tikzpicture}
\end{center}
\begin{verbatim}
\begin{tikzpicture}
\gdef\yoff{1.5}
\gdef\xoff{1.5}
\neuron{minimum size=10pt,fill=orange!50}
\nninput{nI}[2][label]
\neuron{minimum size=15pt,fill=yellow!50}
\nnhidden{nH1}[3][1]
\neuron{minimum size=25pt,fill=gray!50}
\nnoutput{nO}[1][2]
\nnfull{nI}{nH1}
\nnfull{nH1}{nO}
\end{tikzpicture}
\end{verbatim}
\vspace{1cm}
% ---------------------------------------------------------------------
\begin{center}
\begin{tikzpicture}
\gdef\yoff{1}
\gdef\xoff{1.5}
\neuron{minimum size=15pt}
\nninput{nI}[4]
\nnhidden{nH1}[2][1]
\nnhidden{nH2}[6][2]
\nnhidden{nH3}[1][3]
\nnhidden{nH4}[2][4]
\nnoutput{nO}[4][5]
\nnfull{nI}{nH1}
\nnfull{nH1}{nH2}
\nnfull{nH2}{nH3}
\nnfull{nH3}{nH4}
\nnfull{nH4}{nO}
\end{tikzpicture}
\end{center}
\begin{verbatim}
\begin{tikzpicture}
\gdef\yoff{1}
\gdef\xoff{1.5}
\neuron{minimum size=15pt}
\nninput{nI}[4]
\nnhidden{nH1}[2][1]
\nnhidden{nH2}[6][2]
\nnhidden{nH3}[1][3]
\nnhidden{nH4}[2][4]
\nnoutput{nO}[4][5]
\nnfull{nI}{nH1}
\nnfull{nH1}{nH2}
\nnfull{nH2}{nH3}
\nnfull{nH3}{nH4}
\nnfull{nH4}{nO}
\end{tikzpicture}
\end{verbatim}
\vspace{1cm}
% ---------------------------------------------------------------------
\begin{center}
\begin{tikzpicture}
\gdef\yoff{1}
\gdef\xoff{1.5}
\neuron{minimum size=15pt}
\nninput{nI}[8]
\nnhidden{nH1}[6][1]
\nnhidden{nH2}[4][2]
\nnhidden{nH3}[2][3]
\nnhidden{nH4}[1][4]
\nnhidden{nH5}[2][5]
\nnhidden{nH6}[4][6]
\nnhidden{nH7}[6][7]
\nnoutput{nO}[8][8]
\nnfull{nI}{nH1}
\nnfull{nH1}{nH2}
\nnfull{nH2}{nH3}
\nnfull{nH3}{nH4}
\nnfull{nH4}{nH5}
\nnfull{nH5}{nH6}
\nnfull{nH6}{nH7}
\nnfull{nH7}{nO}
\end{tikzpicture}
\end{center}
\begin{verbatim}
\begin{tikzpicture}
\gdef\yoff{1}
\gdef\xoff{1.5}
\neuron{minimum size=15pt}
\nninput{nI}[8]
\nnhidden{nH1}[6][1]
\nnhidden{nH2}[4][2]
\nnhidden{nH3}[2][3]
\nnhidden{nH4}[1][4]
\nnhidden{nH5}[2][5]
\nnhidden{nH6}[4][6]
\nnhidden{nH7}[6][7]
\nnoutput{nO}[8][8]
\nnfull{nI}{nH1}
\nnfull{nH1}{nH2}
\nnfull{nH2}{nH3}
\nnfull{nH3}{nH4}
\nnfull{nH4}{nH5}
\nnfull{nH5}{nH6}
\nnfull{nH6}{nH7}
\nnfull{nH7}{nO}
\end{tikzpicture}
\end{verbatim}
\vspace{1cm}
% ---------------------------------------------------------------------
\begin{center}
\begin{tikzpicture}
\gdef\yoff{0.5}
\gdef\xoff{2}
\neuron{minimum size=10pt}
\nninput{nI}[32]
\nnhidden{nH1}[10][2]
\nnoutput{nO}[10][3]
\nnfull{nI}{nH1}{ultra thin}
\nnfull{nH1}{nO}{ultra thick}
\end{tikzpicture}
\end{center}
\begin{verbatim}
\begin{tikzpicture}
\gdef\yoff{0.5}
\gdef\xoff{2}
\neuron{minimum size=10pt}
\nninput{nI}[32]
\nnhidden{nH1}[10][2]
\nnoutput{nO}[10][3]
\nnfull{nI}{nH1}{ultra thin}
\nnfull{nH1}{nO}{ultra thick}
\end{tikzpicture}
\end{verbatim}
% ---------------------------------------------------------------------

\section{Code des macros}
\begin{verbatim}
\NeedsTeXFormat{LaTeX2e}[1999/01/01]
\ProvidesPackage{nntikz}[2018/09/25]

\RequirePackage{ifthen}
\RequirePackage{xargs}
\RequirePackage{xparse}
\RequirePackage{tikz}
\usetikzlibrary{calc,shapes,arrows}
\usetikzlibrary{decorations.pathreplacing,decorations.markings}

% macros tracers des graphes de réseau de neurones
% version 1.0

\gdef\xoff{1}
\gdef\yoff{1}
\tikzstyle{neuron}=[thick,draw,circle,minimum size=10pt,inner sep=0pt]
\newcommand{\neuron}[1]{
\tikzstyle{neuron}+=[#1]
}
\tikzstyle{neuron input}=[fill=green!50,neuron]
\tikzstyle{neuron output}=[fill=red!50,neuron]

\newcommandx{\nninput}[3][2=1,3=]
{
	\def\dim{#2}
        \expandafter\newcommand\csname#1\endcsname{#2}
        \foreach \xn in {1,...,\dim} {
		\pgfmathsetmacro{\y}{\yoff*\xn-(\dim-1)*\yoff/2}
                \ifthenelse{\equal{#3}{label}}{
                    \node[minimum size=20pt,neuron input] (#1-\xn) at (0,\y) {$x_{\xn}$};
                }{
                    \node[neuron input] (#1-\xn) at (0,\y) {};
                }
	    }
}
\newcommandx{\nnoutput}[4][2=1,3=0,4=]
{
	\def\dim{#2}
        \expandafter\newcommand\csname#1\endcsname{#2}
	\foreach \xn in {1,...,\dim} {
		\pgfmathsetmacro{\y}{\yoff*\xn-(\dim-1)*\yoff/2}
                \ifthenelse{\equal{#4}{label}}{
                    \node[minimum size=20pt,neuron output] (#1-\xn) at (#3*\xoff,\y) {$y_{\xn}$};
                }{
                    \node[neuron output] (#1-\xn) at (#3*\xoff,\y) {};
                }
	    }
}
\newcommandx{\nnhidden}[4][2=1,3=0,4=]
{
	\def\dim{#2}
	\pgfmathsetmacro{\colo}{#3*15}
        \expandafter\newcommand\csname#1\endcsname{#2}
	\foreach \xn in {1,...,\dim} {
		\pgfmathsetmacro{\y}{\yoff*\xn-(\dim-1)*\yoff/2}
                \ifthenelse{\equal{#4}{label}}{
                    \node[minimum size=20pt,fill=blue!\colo,neuron] (#1-\xn) at (#3*\xoff,\y) {$h_{#3\xn}$};
                }{
                    \node[fill=blue!\colo,neuron] (#1-\xn) at (#3*\xoff,\y) {};
                }
	    }
}

\newcommandx{\nnfull}[3][3=]{%
	\foreach \source in {1,...,\csname#1\endcsname}{
		\foreach \dest in {1,...,\csname#2\endcsname}
		{
			\path (#1-\source) edge[-latex,#3] (#2-\dest);
		}
	}
}
\end{verbatim}
\end{document}

