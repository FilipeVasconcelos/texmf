\documentclass[a4paper,9pt]{article}
\usepackage[utf8]{inputenc}
\usepackage[T1]{fontenc}
\usepackage[francais]{babel}
\usepackage{lmodern}

\usepackage{titling}
\usepackage{tikz}
\usepackage{xargs}
\usepackage{etoolbox}% http://ctan.org/pkg/etoolbox
\usepackage{nntikz}
\usepackage{verbatim}

\newcommand{\subtitle}[1]{%
      \posttitle{%
              \par\end{center}
                  \begin{center}\large#1\end{center}
                          \vskip0.5em}%
                          }
\newcommand{\mparagraph}[1]{\paragraph{#1}\mbox{}\\}

\title{Réseau de neurones avec PGF/Tikz}
\subtitle{nntikz version 1.0}
\author{F. M. Vasconcelos}
\date{}

\begin{document}
\maketitle
\begin{abstract}
Macros tikz pour tracer des graphes de réseau de neurones. 
\end{abstract}


%%%%%%%%%%%%%%%%%%%%%%%%%%%%%%%%%%%%%%%%%%%%%%%%%%%%%%%%%%%%%%%%%%%%%%%%%%%%%%
\section{Paquets nécessaires}
%%%%%%%%%%%%%%%%%%%%%%%%%%%%%%%%%%%%%%%%%%%%%%%%%%%%%%%%%%%%%%%%%%%%%%%%%%%%%%

%%%%%%%%%%%%%%%%%%%%%%%%%%%%%%%%%%%%%%%%%%%%%%%%%%%%%%%%%%%%%%%%%%%%%%%%%%%%%%
\section{Utilisation des macros}
%%%%%%%%%%%%%%%%%%%%%%%%%%%%%%%%%%%%%%%%%%%%%%%%%%%%%%%%%%%%%%%%%%%%%%%%%%%%%%

\begin{center}
\begin{tikzpicture}
\nninput{I}[4]
\nnhidden{H1}[3][1.5]
\nnhidden{H2}[3][3]
\nnhidden{H3}[4][4.5]
\nnhidden{H4}[3][6]
\nnoutput{O}[3][7.5]
%\nnfull{I}{3}{H1}{4}
\nnfull{I}{H1}
\nnfull{H1}{H2}
\nnfull{H2}{H3}
\nnfull{H3}{H4}
\nnfull{H4}{O}
\end{tikzpicture}
\end{center}

\end{document}

