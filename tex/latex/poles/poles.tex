\documentclass[a4paper,9pt]{article}
\usepackage[utf8]{inputenc}
\usepackage[T1]{fontenc}
\usepackage[francais]{babel}
\usepackage{lmodern}
\usepackage{titling}


\usepackage{amsmath}
\usepackage{tikz}
\usetikzlibrary{external}
\usepackage{poles}

\newcommand{\subtitle}[1]{%                          
\posttitle{%                                   
\par\end{center}%
\begin{center}\large#1\end{center}%
\vskip0.5em}%              
}

\title{Macros pour la création de cartes de pôles et zéros}
\subtitle{\texttt{poles} version 1.0}
\author{F. M. Vasconcelos}
\date{}

\begin{document}
\maketitle
\begin{abstract}
    Macros pour la création de cartes de pôles et zéros sous PGF/TikZ.
\end{abstract}

\section{Usage}


\begin{tikzpicture}
        \carte[l][2.0]
        \dpole{-0.5}{1}[1][green]
        \dpole{-0.5}{-1}[2][green]
        \dzero{-0.5}{0}[1][magenta]
        \dzero{-1.5}{-1}[2][magenta]
\end{tikzpicture}

\begin{tikzpicture}
    \carte[c][1.0]
        \dpole{-0.5}{1}[1][green]
        \dpole{-0.5}{-1}[2][green]
        \dzero{-0.5}{0}[1][magenta]
        \dzero{-1.5}{-1}[2][magenta]
\end{tikzpicture}

\begin{tikzpicture}
    \carte[r][0.0]
        \dpole{1.5}{1}[1][blue]
        \dpole{1.5}{-1}[2][blue]
        \dzero{-0.5}{0}[1][red]
        \dzero{1.0}{0}[2][red]
\end{tikzpicture}



\end{document}
