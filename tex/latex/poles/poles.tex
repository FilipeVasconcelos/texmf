\documentclass[a4paper,9pt]{article}
\usepackage[utf8]{inputenc}
\usepackage[T1]{fontenc}
\usepackage[francais]{babel}
\usepackage{lmodern}
\usepackage{titling}


\usepackage{amsmath}
\usepackage{tikz}
\usetikzlibrary{external}
\usetikzlibrary{calc}
\usepackage{poles}

\newcommand{\subtitle}[1]{%                          
\posttitle{%                                   
\par\end{center}%
\begin{center}\large#1\end{center}%
\vskip0.5em}%              
}

\title{Macros pour la création de cartes de pôles et zéros}
\subtitle{\texttt{poles} version 2.0}
\author{F. M. Vasconcelos}
\date{}

\begin{document}
\maketitle
\begin{abstract}
    Macros pour la création de cartes de pôles et zéros sous PGF/TikZ.
\end{abstract}

\section{Usage}

\begin{center}
\begin{tikzpicture}[scale=1.2]
    \carte[l]
    \dpole{-1.5}{1}{1}[orange]
    \dpole{-1.5}{-1}{2}[orange]
    \dzero{0}{0}{1}[cyan]
    \dzero{-1}{0}{2}[cyan]
\end{tikzpicture}
\end{center}

Commande:
\begin{verbatim}
\begin{tikzpicture}[scale=1.2]
    %axis  l=left
    \carte[l]
    %        x   y index color
    \dpole{-1.5}{1}[1][orange]
    \dpole{-1.5}{-1}[2][orange]
    \dzero{0}{0}[1][cyan]
    \dzero{-1}{0}[2][cyan]
\end{tikzpicture}
\end{verbatim}

\begin{center}
\begin{tikzpicture}[scale=1.2]
    \carte[c]
    \dpole{0.0}{1}{1}[green][left] 
    \dpole{0.0}{-1}{2}[green][left]
    \dzero{-0.5}{0}{1}[magenta][above]
    \dzero{1.0}{0}{2}[magenta][above]
\end{tikzpicture}
\end{center}

Commande:
\begin{verbatim}
\begin{tikzpicture}[scale=1.2]
    % axis c=centered
    \carte[c]
    %        x  y index color pos
    \dpole{0.0}{1}[1][green][left] 
    \dpole{0.0}{-1}[2][green][left]
    \dzero{-0.5}{0}[1][magenta][above]
    \dzero{1.0}{0}[2][magenta][above]
\end{tikzpicture}
\end{verbatim}

\begin{center}
\begin{tikzpicture}[scale=1.2]
    \carte[r]
    \dpole{1.5}{1}{1}[blue][right]
    \tics{1}[p]
    \dpole{1.5}{-1}{2}[blue][right]
    \tics{2}[p]
    \dzero{-0.5}{0}{1}[red]
    \tics{1}[z]
    \dzero{1.0}{0}{2}[red]
    \tics{2}[z]
\end{tikzpicture}
\end{center}

Commande:
\begin{verbatim}
\begin{tikzpicture}[scale=1.2]
    %axis  r=right
    \carte[r]
    \dpole{1.5}{1}[1][blue][right]
    \tics{1}[p]
    \dpole{1.5}{-1}[2][blue][right]
    \tics{2}[p]
    \dzero{-0.5}{0}[1][red]
    \tics{1}[z]
    \dzero{1.0}{0}[2][red]
    \tics{2}[z]
\end{tikzpicture}
\end{verbatim}

\begin{center}
\begin{tikzpicture}[scale=1.2]
    \carte[l][-5][5][-2][2][3.0]
    \dpole{-1.5}{1}{1}[orange]
    \tics{1}[p]
    \dpole{-1.5}{-1}{2}[orange]
    \tics{2}[p]
    \dzero{0}{0}{1}[cyan][above]
    \tics{1}[z]
    \dzero{-1}{0}{2}[cyan][above]
    \tics{2}[z]
\end{tikzpicture}
\end{center}

Commande:
\begin{verbatim}
\begin{tikzpicture}[scale=1.2]
    \carte[l][-5][5][-2][2][3.0]
    \dpole{-1.5}{1}{1}[orange]
    \tics{1}[p]
    \dpole{-1.5}{-1}{2}[orange]
    \tics{2}[p]
    \dzero{0}{0}{1}[cyan]
    \tics{1}[z]
    \dzero{-1}{0}{2}[cyan]
    \tics{2}[z]
\end{tikzpicture}
\end{verbatim}

\end{document}
