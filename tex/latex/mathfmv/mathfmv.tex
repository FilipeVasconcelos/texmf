\documentclass[a4paper,9pt]{article} 
\usepackage[utf8]{inputenc}          
\usepackage[T1]{fontenc}             
\usepackage[francais]{babel}         
\usepackage{lmodern}                 
\usepackage{geometry}                                                                                                         
\geometry{hmargin=2.5cm,vmargin=2.6cm}
\usepackage{amsmath}
\usepackage{mathrsfs}
\usepackage{mathfmv}
\usepackage{vecteurs}
\usepackage{verbatim}

\begin{document}

\begin{verbatim}
% systèmes de coordonnnées pour tikz
\newcommand{\ptcyl}[3]{{#1*cos(#2)},{#1*sin(#2)},{#3}}
\newcommand{\ptsph}[3]{{#1*sin(#2)*cos(#3)},{#1*sin(#2)*sin(#3)},{#1*cos(#2)}}
\end{verbatim}
\[
    \ptcyl{0.1}{0.2}{0.4}
\]
\[
    \ptsph{0.3}{0.4}{0.2}
\]

\begin{verbatim}
% Laplace 
\newcommand*{\laplace}[1]{\mathscr{L}\left\{#1\right\}}
\end{verbatim}
\[
    \laplace{s}
\]
\begin{verbatim}
% Inverse Laplace
\newcommand*{\laplacei}[1]{\mathscr{L}^{-1}\left\{#1\right\}}
\end{verbatim}
\[
    \laplacei{F}
\]
\begin{verbatim}
\newcommand*{\rttensor}[1]{\overline{\overline{#1}}}
\end{verbatim}
\[
    \rttensor{A}
\]

\begin{verbatim}
% Complex numbers
\renewcommand{\Re}[1]{\mathrm{Re}[#1]}
\renewcommand{\Im}[1]{\mathrm{Im}[#1]}
% j omega
\newcommand*{\jw}{j\omega}
\newcommand*{\jtw}{j\tau\omega}
\newcommand*{\tw}{\tau\omega}
\newcommand*{\ttww}{\tau^2\omega^2}
\end{verbatim}
\[
    \Re{p},\Im{p}
\]
\[
    \jw,\jtw,\tw,\ttww
\]
\begin{verbatim}
% d droit dérivée
\newcommand*{\dd}[1]{\mathrm{d}{#1}}
\newcommand*{\ddd}[1]{\mathrm{d^2}{#1}}
\newcommand*{\devi}[2]{\dfrac{\mathrm{d}^{#2}{#1}}{\dd{t^{#2}}}}
\newcommand*{\deviV}[3]{\left[\devi{#1}{#2}\right]_{#3}}
\newcommand*{\deviR}[3]{\dfrac{\mathrm{d}^{#2}{#1}}{\dd{#3^{#2}}}}
\end{verbatim}
\[
    \dd{x},\dd{y}
\]
\[
    \ddd{x},\ddd{y}
\]
\[
    \devi{x}{},\devi{y}{3}
\]
\[
    \deviV{\xx{3}}{}{R_0}
\]
\[
    \deviR{f(y)}{2}{y}
\]
\clearpage
\begin{verbatim}
% cos/sin
\newcommandx{\cosa}[1][1=]{\cos\alpha_{#1}}
\newcommandx{\cosb}[1][1=]{\cos\alpha_{#1}}
\newcommandx{\sina}[1][1=]{\sin\alpha_{#1}}
\newcommandx{\sinb}[1][1=]{\sin\alpha_{#1}}
\end{verbatim}
\[
    \cosa[1],\cosa[2],\sina[1],\sinb[2]
\]

\begin{verbatim}
% dot angle
\newcommand*{\dalpha}{\dot{\alpha}}
\newcommand*{\dtheta}{\dot{\theta}}
\newcommand*{\dphi}{\dot{\phi}}
\newcommand*{\dbeta}{\dot{\beta}}
\newcommand*{\dpsi}{\dot{\psi}}
\newcommand*{\drho}{\dot{\rho}}
\newcommand*{\dmu}{\dot{\mu}}
\end{verbatim}
\[
    \dalpha,\dtheta,\dphi,\dbeta,\dpsi,\drho,\dmu
\]
\begin{verbatim}
% ddot angle
\newcommand*{\ddalpha}{\ddot{\alpha}}
\newcommand*{\ddtheta}{\ddot{\theta}}
\newcommand*{\ddphi}{\ddot{\phi}}
\newcommand*{\ddbeta}{\ddot{\beta}}
\newcommand*{\ddpsi}{\ddot{\psi}}
\end{verbatim}
\[
    \ddalpha,\ddtheta,\ddphi,\ddbeta,\ddpsi
\]

\begin{verbatim}
% dot longueur
\newcommand*{\dlambda}{\dot{\lambda}}
\end{verbatim}
\[
    \dlambda
\]

\begin{verbatim}
%quelques boldsymbol
\newcommand{\bdM}{\boldsymbol{M}}
\newcommand{\bdI}{\boldsymbol{I}}
\newcommand{\bdA}{\boldsymbol{A}}
\newcommand{\bdB}{\boldsymbol{B}}
\newcommand{\bdC}{\boldsymbol{C}}
\end{verbatim}
\[
    \bdM,\bdI,\bdA,\bdB,\bdC
\]

\end{document}
