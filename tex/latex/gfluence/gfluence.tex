\documentclass[a4paper,9pt]{article}
\usepackage[utf8]{inputenc}
\usepackage[T1]{fontenc}
\usepackage[francais]{babel}
\usepackage{lmodern}

\usepackage{titling}
\usepackage{pgf,tikz}
\usepackage{pgfplots}
\pgfplotsset{compat=newest}
\usetikzlibrary{calc,shapes,arrows,decorations.text}
\usetikzlibrary{decorations.pathreplacing,decorations.markings}
\usepackage{xargs}
\usepackage{gfluence}
\usepackage{verbatim}

\newcommand{\subtitle}[1]{%
      \posttitle{%
              \par\end{center}
                  \begin{center}\large#1\end{center}
                          \vskip0.5em}%
                          }
\newcommand{\mparagraph}[1]{\paragraph{#1}\mbox{}\\}

\title{Graphe de fluence avec PGF/Tikz}
\subtitle{gfluence version 1.0}
\author{F. M. Vasconcelos}
\date{}

\begin{document}
\maketitle
\begin{abstract}
Macros tikz pour tracer des graphes de fluence pour l’automatique
comme présentés dans Ostertag~\cite{Ostertag}. Le paquet est largement inspiré du
paquet schemabloc de R. Papanicola ~\cite{schemabloc}.
\end{abstract}


%%%%%%%%%%%%%%%%%%%%%%%%%%%%%%%%%%%%%%%%%%%%%%%%%%%%%%%%%%%%%%%%%%%%%%%%%%%%%%%%%
\section{Paquets nécessaires}
%%%%%%%%%%%%%%%%%%%%%%%%%%%%%%%%%%%%%%%%%%%%%%%%%%%%%%%%%%%%%%%%%%%%%%%%%%%%%%%%%
\begin{verbatim}
\usepackage{tikz}
\usepackage{xargs}
\usepackage{xparse}
\usepackage{gfluence}
\end{verbatim}
Le paquet \verb?ifthen? est nécessaire mais 
semble être charger par un autre paquet.

%%%%%%%%%%%%%%%%%%%%%%%%%%%%%%%%%%%%%%%%%%%%%%%%%%%%%%%%%%%%%%%%%%%%%%%%%%%%%%%%%
\section{Utilisation des macros}
%%%%%%%%%%%%%%%%%%%%%%%%%%%%%%%%%%%%%%%%%%%%%%%%%%%%%%%%%%%%%%%%%%%%%%%%%%%%%%%%%
\mparagraph{Définition des noeuds d'entrée et de sortie}

\begin{center}
\begin{tikzpicture}
    \gfEntree{$E$}{E}
    \gfSortie{$S$}{E}{S}
\end{tikzpicture}
\end{center}
\begin{verbatim}
\begin{tikzpicture}
    \gfEntree{$E$}{E}
    \gfSortie{$S$}{E}{S}
\end{tikzpicture}
\end{verbatim}

\mparagraph{Simple lien entre les noeuds}

\begin{center}
\begin{tikzpicture}
    \gfEntree{$E$}{E}
    \gfSortie{$S$}{E}{S}
    \gfRelier{E}{S}
\end{tikzpicture}
\end{center}
\begin{verbatim}
\begin{tikzpicture}
    \gfEntree{$E$}{E}
    \gfSortie{$S$}{E}{S}
    \gfRelier{E}{S}
\end{tikzpicture}
\end{verbatim}

\mparagraph{Simple lien entre deux noeuds : options}

\begin{center}
\begin{tikzpicture}
    \gfEntree{$E$}{E}
    \gfSortie[12]{$S$}{E}{S}
    \gfRelier[$H$]{E}{S}
\end{tikzpicture}
\end{center}
\begin{verbatim}
\begin{tikzpicture}
    \gfEntree{$E$}{E}
    \gfSortie[12]{$S$}{E}{S}
    \gfRelier[$H$]{E}{S}
\end{tikzpicture}
\end{verbatim}

\begin{center}
\begin{tikzpicture}
    \gfEntree[0][0][]{}{E}
    \gfEntree[0][3][+]{$E_1$}{E1}
    \gfEntree[0][-3][-]{$E_2$}{E2}
    \gfSortie[4]{$S$}{E}{S}
    \gfRelier[$H_1$][+][1.2]{E1}{S}
    \gfRelier[$H_2$][-][1.2]{E2}{S}
\end{tikzpicture}
\end{center}
\begin{verbatim}
\begin{tikzpicture}
    \gfEntree[0][0][]{}{E}
    \gfEntree[0][3][+]{$E_1$}{E1}
    \gfEntree[0][-3][-]{$E_2$}{E2}
    \gfSortie[8]{$S$}{E}{S}
    \gfRelier[$H_1$][+][1.2]{E1}{S}
    \gfRelier[$H_2$][-][1.2]{E2}{S}
\end{tikzpicture}
\end{verbatim}

\mparagraph{Boucle élémentaire}

\begin{center}
\begin{tikzpicture}
    \gfEntree{$E$}{E}
    \gfSortie{$S$}{E}{S}
    \gfRelier[$H$]{E}{S}
    \gfBoucleE{S}
\end{tikzpicture}
\end{center}
\begin{verbatim}
\begin{tikzpicture}
    \gfEntree{E}{$E$}
    \gfSortie{$S$}{E}{S}
    \gfRelier[$H$]{E}{S}
    \gfBoucleE{S}
\end{tikzpicture}
\end{verbatim}
\newpage

\mparagraph{Boucle élémentaire: options}

\begin{center}
\begin{tikzpicture}
    \gfEntree{$E$}{E}
    \gfSortie{$S$}[-]{E}{S}
    \gfRelier[$H$]{E}{S}
    \gfBoucleE[$G$][+]{S}
\end{tikzpicture}
\end{center}
\begin{verbatim}
\begin{tikzpicture}
    \gfEntree{E}{$E$}
    \gfSortie{$S$}{E}{S}
    \gfRelier[$H$]{E}{S}
    \gfBoucleE[$G$][+]{S}
\end{tikzpicture}
\end{verbatim}

\mparagraph{Noeud supplémentaire}

\begin{center}
\begin{tikzpicture}
    \gfEntree{$E$}{E}
    \gfNoeud{$X_1$}{E}{X1}
    \gfSortie{$S$}{X1}{S}
    \gfRelier[$H_1$]{E}{X1}
    \gfRelier[$H_2$]{X1}{S}
    \gfBoucleE[$G$]{S}
\end{tikzpicture}
\end{center}
\begin{verbatim}
\begin{tikzpicture}
    \gfEntree{$E$}{E}
    \gfNoeud{$X_1$}{E}{X1}
    \gfSortie{$S$}{X1}{S}
    \gfRelier[$H_1$]{E}{X1}
    \gfRelier[$H_2$]{X1}{S}
    \gfBoucleE[$G$]{S}
\end{tikzpicture}
\end{verbatim}

\mparagraph{Noeud supplémentaire: options}

\begin{center}
\begin{tikzpicture}
    \gfEntree{$E$}{E}
    \gfNoeud[8]{$X_1$}[-][0.9]{E}{X1}
    \gfSortie{$S$}{X1}{S}
    \gfRelier[$H_1$]{E}{X1}
    \gfRelier[$H_2$]{X1}{S}
    \gfBoucleE[$G$]{S}
\end{tikzpicture}
\end{center}
\begin{verbatim}
\begin{tikzpicture}
    \gfEntree{$E$}{E}
    \gfNoeud[8]{$X_1$}[-][0.9]{E}{X1}
    \gfSortie{$S$}{X1}{S}
    \gfRelier[$H_1$]{E}{X1}
    \gfRelier[$H_2$]{X1}{S}
    \gfBoucleE[$G$]{S}
\end{tikzpicture}
\end{verbatim}

\mparagraph{Lien courbe}

\begin{center}
\begin{tikzpicture}
    \gfEntree{$E$}{E}
    \gfNoeud{$X_1$}{E}{X1}
    \gfNoeud{$X_2$}{X1}{X2}
    \gfSortie{$S$}{X2}{S}
    \gfRelier[$H_1$]{E}{X1}
    \gfRelier[$H_2$]{X1}{X2}
    \gfRelier[$H_3$]{X2}{S}
    \gfRelierB{X1}{X2}
    \gfBoucleE[$G$]{S}
\end{tikzpicture}
\end{center}
\begin{verbatim}
\begin{tikzpicture}
    \gfEntree{$E$}{E}
    \gfNoeud{$X_1$}{E}{X1}
    \gfNoeud{$X_2$}{X1}{X2}
    \gfSortie{$S$}{X2}{S}
    \gfRelier[$H_1$]{E}{X1}
    \gfRelier[$H_2$]{X1}{X2}
    \gfRelier[$H_3$]{X2}{S}
    \gfRelierB{X1}{X2}
    \gfBoucleE[$G$]{S}
\end{tikzpicture}
\end{verbatim}

\mparagraph{Lien courbe: options}

\begin{center}
\begin{tikzpicture}
    \gfEntree{$E$}{E}
    \gfNoeud{$X_1$}[-]{E}{X1}
    \gfNoeud{$X_2$}[-]{X1}{X2}
    \gfSortie{$S$}{X2}{S}
    \gfRelier[$H_1$]{E}{X1}
    \gfRelier[$H_2$]{X1}{X2}
    \gfRelier[$H_3$]{X2}{S}
    \gfRelierB[$G_1$][+]{X1}{X2}
    \gfBoucleE[$G$]{S}
\end{tikzpicture}
\end{center}
\begin{verbatim}
\begin{tikzpicture}
    \gfEntree{$E$}{E}
    \gfNoeud{$X_1$}[-]{E}{X1}
    \gfNoeud{$X_2$}[-]{X1}{X2}
    \gfSortie{$S$}{X2}{S}
    \gfRelier[$H_1$]{E}{X1}
    \gfRelier[$H_2$]{X1}{X2}
    \gfRelier[$H_3$]{X2}{S}
    \gfRelierB[$G_1$][+]{X1}{X2}
    \gfBoucleE[$G$]{S}
\end{tikzpicture}
\end{verbatim}

\begin{center}
\begin{tikzpicture}
    \gfEntree{$E$}{E}
    \gfNoeud{$X_1$}[-]{E}{X1}
    \gfNoeud{$X_2$}[-]{X1}{X2}
    \gfSortie{$S$}{X2}{S}
    \gfRelier[$H_1$]{E}{X1}
    \gfRelier[$H_2$]{X1}{X2}
    \gfRelier[$H_3$]{X2}{S}
    \gfRelierB[$G_1$][+]{X1}{X2}
    \gfRelierB[$G_2$][+]{X1}{S}
    \gfBoucleE[$G$]{S}
\end{tikzpicture}
\end{center}
\begin{verbatim}
\begin{tikzpicture}
    \gfEntree{$E$}{E}
    \gfNoeud{$X_1$}[-][-0.7]{E}{X1}
    \gfNoeud{$X_2$}[-][0.7]{X1}{X2}
    \gfSortie{$S$}{X2}{S}
    \gfRelier[$H_1$]{E}{X1}
    \gfRelier[$H_2$]{X1}{X2}
    \gfRelier[$H_3$]{X2}{S}
    \gfRelierB[$G_1$][+]{X1}{X2}
    \gfRelierB[$G_2$][+]{X1}{S}
    \gfBoucleE[$G$]{S}
\end{tikzpicture}
\end{verbatim}


\begin{center}
\begin{tikzpicture}
    \gfEntree{$E$}{E}
    \gfNoeud{$X_1$}[-][-0.7]{E}{X1}
    \gfNoeud{$X_2$}[-][0.7]{X1}{X2}
    \gfSortie{$S$}{X2}{S}
    \gfRelier[$H_1$]{E}{X1}
    \gfRelier[$H_2$]{X1}{X2}
    \gfRelier[$H_3$]{X2}{S}
    \gfRelierB[$G_1$][-]{X1}{X2}
    \gfRelierB[$G_2$][+]{X1}{S}
    \gfBoucleE[$G$]{S}
\end{tikzpicture}
\end{center}
\begin{verbatim}
\begin{tikzpicture}
    \gfEntree{$E$}{E}
    \gfNoeud{$X_1$}[-][-0.7]{E}{X1}
    \gfNoeud{$X_2$}[-][0.7]{X1}{X2}
    \gfSortie{$S$}{X2}{S}
    \gfRelier[$H_1$]{E}{X1}
    \gfRelier[$H_2$]{X1}{X2}
    \gfRelier[$H_3$]{X2}{S}
    \gfRelierB[$G_1$][-]{X1}{X2}
    \gfRelierB[$G_2$][+]{X1}{S}
    \gfBoucleE[$G$]{S}
\end{tikzpicture}
\end{verbatim}

\newpage
%%%%%%%%%%%%%%%%%%%%%%%%%%%%%%%%%%%%%%%%%%%%%%%%%%%%%%%%%%%%%%%%%%%%%%%%%%%%%%%%%
\section{Exemple de grande taille}
%%%%%%%%%%%%%%%%%%%%%%%%%%%%%%%%%%%%%%%%%%%%%%%%%%%%%%%%%%%%%%%%%%%%%%%%%%%%%%%%%

\begin{center}
\begin{tikzpicture}
    \gfEntree{$E$}{E}
    \gfNoeud{$X_1$}[+][-0.7]{E}{X1}
    \gfNoeud{$X_2$}[+][-0.7]{X1}{X2}
    \gfNoeud{$X_3$}[-][0.7]{X2}{X3}
    \gfNoeud{$X_4$}[-]{X3}{X4}
    \gfSortie{$S$}{X4}{S}
    \gfRelier[$H_1$]{E}{X1}
    \gfRelier[$H_2$]{X1}{X2}
    \gfRelier[$H_3$]{X2}{X3}
    \gfRelier[$H_4$]{X3}{X4}
    \gfRelier[$H_5$]{X4}{S}
    \gfRelierB[$G_1$][-]{X1}{X2}
    \gfRelierB[$G_2$][+]{X2}{X4}
    \gfRelierB[$G_2$][-]{X2}{X3}
    \gfRelierB[$G_2$][+]{X1}{S}
    \gfBoucleE[$G_1$]{S}
    \gfBoucleE[$G_2$][-][-1.7]{X1}
\end{tikzpicture}
\end{center}
\begin{verbatim}
\begin{tikzpicture}
    \gfEntree{$E$}{E}
    \gfNoeud{$X_1$}[+][-0.7]{E}{X1}
    \gfNoeud{$X_2$}[+][-0.7]{X1}{X2}
    \gfNoeud{$X_3$}[-][0.7]{X2}{X3}
    \gfNoeud{$X_4$}[-][0.7]{X3}{X4}
    \gfSortie{$S$}{X4}{S}
    \gfRelier[$H_1$]{E}{X1}
    \gfRelier[$H_2$]{X1}{X2}
    \gfRelier[$H_3$]{X2}{X3}
    \gfRelier[$H_4$]{X3}{X4}
    \gfRelier[$H_5$]{X4}{S}
    \gfRelierB[$G_1$][-]{X1}{X2}
    \gfRelierB[$G_2$][+]{X2}{X4}
    \gfRelierB[$G_2$][-]{X2}{X3}
    \gfRelierB[$G_2$][+]{X1}{S}
    \gfBoucleE[$G$]{S}
    \gfBoucleE[$G$]{X1}
\end{tikzpicture}
\end{verbatim}

\newpage
%%%%%%%%%%%%%%%%%%%%%%%%%%%%%%%%%%%%%%%%%%%%%%%%%%%%%%%%%%%%%%%%%%%%%%%%%%%%%%%%%
\section{Code des macros}
%%%%%%%%%%%%%%%%%%%%%%%%%%%%%%%%%%%%%%%%%%%%%%%%%%%%%%%%%%%%%%%%%%%%%%%%%%%%%%%%%
\begin{verbatim}
\NeedsTeXFormat{LaTeX2e}[1999/01/01]
\ProvidesPackage{gfluence}[2018/06/25]

\RequirePackage{ifthen}
\RequirePackage{xparse}
\RequirePackage{tikz}
\usetikzlibrary{calc,shapes,arrows}
\usetikzlibrary{decorations.pathreplacing,decorations.markings}

%macros dessin de graphe de fluence comme présenté dans Ostertag
% version 1.0

\tikzset{
  % style to add an arrow in the middle of a path
  mid arrow/.style={postaction={decorate,decoration={
        markings,
        mark=at position .5 with {\arrow[scale=1.8,#1]{stealth}}
        }}},
}

%initialisation des styles
\tikzstyle{gfStyleLien}=[thick,postaction={mid arrow=black}]
\tikzstyle{gfStyleNoeud}=[thick,draw, circle,inner sep=2pt,minimum size=0.1em]

\newcommand{\gfStyleLien}[1]{
\tikzstyle{gfStyleLien}+=[#1]
}

\newcommand{\gfStyleNoeud}[1]{
\tikzstyle{gfStyleNoeud}+=[#1]
}

% Commandes d'entrée et sortie
\newcommandx{\gfEntree}[5][1=0,2=0,3=+]{
    \node (#5O) at (0,#2em) {};
    \node [right of=#5O, node distance=#1em,name=#5,coordinate] {};
    \gfDecaleNoeudx[0]{#5}{#5}
    \ifthenelse{\equal{#3}{}}{
    }{
        \node [gfStyleNoeud,name=#5] at (#5) {};
    }
    \ifthenelse{\equal{#3}{+}}{
    \node [xshift=-0.8em,yshift=0.8em] at (#5) {#4};
    }{
    \node [xshift=-0.8em,yshift=-0.8em] at (#5) {#4};
    }
}
\newcommandx{\gfSortie}[5][1=6,3=+]{
    \node [gfStyleNoeud, right of=#4droite, node distance=#1em] (#5) {};
    \ifthenelse{\equal{#3}{+}}{
    \node [xshift=0.8em,yshift=0.8em] at (#5) {#2};
    }{
    \node [xshift=0.8em,yshift=-0.8em] at (#5) {#2};
    }
}

\newcommandx{\gfNoeud}[6][1=6,3=+,4=0]{
    \node [right of=#5droite, node distance=#1em,name=#6,coordinate] {};
    \gfDecaleNoeudx[0]{#6}{#6}
    \node [gfStyleNoeud,name=#2,right of=#5droite,node distance=#1em] (#6) {};
    \ifthenelse{\equal{#3}{+}}{
        \node [yshift=0.8em,xshift=#4em] at (#6) {#2};
    }{
        \node [yshift=-0.8em,xshift=#4em] at (#6) {#2};
    }
}
%boucle élémentaire
\newcommandx{\gfBoucleE}[5][1=,2=-,3=2.2,4=1]{
    \ifthenelse{\equal{#2}{-}}{
        \node [below of=#5, node distance=1.1em, minimum size=0em](NO) {};
        \draw[gfStyleLien] (NO) +(75:0.4cm) arc(75:-255:0.4cm) 
        node[xshift=#3em,yshift=-#4em] {#1};
    }{
        \node [above of=#5, node distance=1.1em, minimum size=0em](NO) {};
        \draw[gfStyleLien] (NO) +(-75:0.4cm) arc(-75:255:0.4cm) 
        node[xshift=#3em,yshift=#4em] {#1};
    }
}

%\Commande de lien
\newcommandx{\gfRelier}[5][1=,2=+,3=0.8]{
\draw [gfStyleLien] (#4) -- node[yshift=#2#3em] {#1} (#5);
}
%\Commande de lien bend
\newcommandx{\gfRelierB}[4][1=,2=-]{
    \ifthenelse{\equal{#2}{+}}{
        \draw [gfStyleLien] (#3) to[out=90,in=90] node[above] {#1} (#4);
    }{
        \draw [gfStyleLien] (#3) to[out=-90,in=-90] node[below] {#1} (#4);
    }
}

%Commandes de décalage de noeud
\newcommand{\gfDecaleNoeudy}[3][5]{
\node [below of=#2, node distance=#1em, minimum size=0em](#3) {};
\node (#3droite) at (#3){};
\node (#3gauche) at (#3){};
}
\newcommand{\gfDecaleNoeudx}[3][5]{
\node [right of=#2, node distance=#1em, minimum size=0em](#3) {};
\node (#3droite) at (#3){};
\node (#3gauche) at (#3){};
}
\end{verbatim}

\bibliographystyle{plain}
\bibliography{gfluence.bib}


\end{document}

