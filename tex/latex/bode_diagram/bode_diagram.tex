\documentclass[a4paper,11pt]{article}
\usepackage[utf8]{inputenc}
\usepackage[T1]{fontenc}
\usepackage[frenchb]{babel}
\usepackage{lmodern}

\usepackage{titling}
\usepackage{xargs}
%\usepackage{amsmath}
%\usepackage{amsthm}
%\usepackage{amssymb}
\usepackage{pgf, tikz}
\usepackage{pgfplots}
\pgfplotsset{compat=1.14}
\usepackage{bode_diagram}

\newcommand{\subtitle}[1]{%
      \posttitle{%
      \par\end{center}
      \begin{center}\large#1\end{center}
      \vskip0.5em}%
}
\newcommand{\mparagraph}[1]{\paragraph{#1}\mbox{}\\}

\title{Diagramme de bode avec PGF/Tikz}
\subtitle{bode\_diagram version 1.0}
\author{F. M. Vasconcelos}
\date{}

\begin{document}
\maketitle
\begin{abstract}
Macros tikz pour tracer des diagramme de Bode. 
\end{abstract}

\begin{center}
\begin{tikzpicture}
\bodegain{2}{1}{0.25}
\bodephase{2}{1}{0.25}
\end{tikzpicture}
\end{center}

\begin{center}
\begin{tikzpicture}
\bodegain{4}{2}{0.25}
\bodephase{4}{2}{0.25}
\end{tikzpicture}
\end{center}

\begin{center}
\begin{tikzpicture}
\bodegain{6}{4}{0.25}
\bodephase{6}{4}{0.25}
\end{tikzpicture}
\end{center}

\begin{center}
\begin{tikzpicture}
\bodegain{10}{8}{0.25}
\bodephase{10}{8}{0.25}
\end{tikzpicture}
\end{center}

\end{document}
