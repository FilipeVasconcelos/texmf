\documentclass[a4paper,9pt]{article} 
\usepackage[utf8]{inputenc}          
\usepackage[T1]{fontenc}             
\usepackage[francais]{babel}         
\usepackage{lmodern}                 
                                     
\usepackage{mymath}
\usepackage{amsmath}
\usepackage{titling}
\usepackage{tikz,pgf}
\usepackage{fig-planes}
\usepackage{verbatim}


\newcommand{\subtitle}[1]{%                          
\posttitle{%                                   
\par\end{center}%
\begin{center}\large#1\end{center}%
\vskip0.5em}%              
}                          
\newcommand{\mparagraph}[1]{\paragraph{#1}\mbox{}\\} 
\title{Figures planes avec PGF/Tikz}
\subtitle{fig-planes version 2.0}
\author{F. M. Vasconcelos}
\date{}

\begin{document}
\maketitle
\begin{abstract}
Macros tikz pour tracer des figures planes pour la mécanique. Les figures planes
permettent de représenter des mouvements de rotation relatifs entre deux repères.
L'ensemble de la figure est paramètable jusqu'au choix des couleurs des repères 
tournant et fixe.
\end{abstract}

%%%%%%%%%%%%%%%%%%%%%%%%%%%%%%%%%%%%%%%%%%%%%%%%%%%%%%%%%%%%%%%%%%%%%%%%%%%%%%%%%     
\section{Paquets nécessaires}                                                         
%%%%%%%%%%%%%%%%%%%%%%%%%%%%%%%%%%%%%%%%%%%%%%%%%%%%%%%%%%%%%%%%%%%%%%%%%%%%%%%%%     
\begin{verbatim}                                                                      
\usepackage{tikz}
\usetikzlibrary{calc,quotes,angles}
\RequirePackage{amsmath}
\RequirePackage{kinematik}
\usepackage{ifthen}
\end{verbatim} 
%%%%%%%%%%%%%%%%%%%%%%%%%%%%%%%%%%%%%%%%%%%%%%%%%%%%%%%%%%%%%%%%%%%%%%%%%%%%%%%%%     
\section{Utilisation des macros}                                                         
%%%%%%%%%%%%%%%%%%%%%%%%%%%%%%%%%%%%%%%%%%%%%%%%%%%%%%%%%%%%%%%%%%%%%%%%%%%%%%%%%     
\subsection*{Accès à l'aide du package}

\figplanes[h]

Commande de l'aide :
\begin{verbatim}
\figplanes[h]
\end{verbatim}
Remarque :
Les balises tikz \texttt{\textbackslash begin\{tikzpicture\}\ldots \textbackslash end\{tikzpicture\}} ne 
sont pas nécessaire dans le mode \texttt{h}

\subsection*{Figure par défaut}

\begin{center}
\begin{tikzpicture}
    \figplanes
\end{tikzpicture}
\end{center}

Commande:
\begin{verbatim}
\begin{tikzpicture}
\figplanes
\end{tikzpicture}
\end{verbatim}

\subsection*{Différents modes}

\paragraph{\texttt{mode=r}}$\,$

\begin{center}
    \begin{tikzpicture}
\figplanes[r]
\end{tikzpicture}
\end{center}

Commande:
\begin{verbatim}
\begin{tikzpicture}
\figplanes[r]
\end{tikzpicture}
\end{verbatim}

\paragraph{\texttt{mode=b}}$\,$

\begin{center}
    \begin{tikzpicture}
\figplanes[b]
\end{tikzpicture}
\end{center}

Commande:
\begin{verbatim}
\begin{tikzpicture}
\figplanes[b]
\end{tikzpicture}
\end{verbatim}

\subsection*{Quelques exemples}

\begin{center}
    \begin{tikzpicture}
\figplanes[n][theta][2][3][red][2][blue]{[x][y][z][A][6]}
\end{tikzpicture}
\end{center}

Commande:
\begin{verbatim}
\begin{tikzpicture}
\figplanes[n][theta][2][3][red][2][blue]{[x][y][z][A][6]}
\end{tikzpicture}
\end{verbatim}

\begin{center}
    \begin{tikzpicture}
\figplanes[b][beta][1][1][green!80][0][orange!80]{[y][z][x][B][4]}
\end{tikzpicture}
\end{center}

Commande:
\begin{verbatim}
\begin{tikzpicture}
\figplanes[b][beta][1][1][green!80][0][orange!80]{[y][z][x][B][4]}
\end{tikzpicture}
\end{verbatim}

\section{Code des macros}
\footnotesize
\begin{verbatim}
% fig-planes package
% 
% (c) Filipe Vasconcelos
%
%% This program can be redistributed and/or modified under the terms
%% of the LaTeX Project Public License Distributed from CTAN archives
%% in directory macros/latex/base/lppl.txt.
% 
\NeedsTeXFormat{LaTeX2e}[1994/06/01]
\ProvidesPackage{fig-planes}
  [2017/06/05 v0.01 LaTeX package dessiner des figures planes;
   2018/09/26 v0.02 LaTeX package dessiner des figures planes en couleur]

\RequirePackage{xargs}
\RequirePackage{xparse}
\RequirePackage{ifthen}
\RequirePackage{mymath}
\RequirePackage{tikz}
\usetikzlibrary{calc,quotes,angles}
\RequirePackage{amsmath}
\RequirePackage{kinematik}
\def\hi{Ce paquet dessine des figures planes tikz de 
changement de base pour la cinématique.}
\let\myDate\date

\newcommandx{\figplanes}[8][1=n,2=alpha,3=,4=1,5=black,6=0,7=black]{ 
\def\mode{#1}
\def\angle{\cang{#2}{#3}}
\def\rt{#4}
\def\ct{#5} % color
\def\rf{#6}
\def\cf{#7} % color
\fplanescontinue#8
}

\newcommandx{\fplanescontinue}[5][1=x,2=y,3=z,4=O,5=]{ 
\def\xbt{\ccvect{#1}[\rt]}
\def\xbf{\ccvect{#1}[\rf]}
\def\ybt{\ccvect{#2}[\rt]}
\def\ybf{\ccvect{#2}[\rf]}
\def\zbt{\ccvect{#3}[\rt]}
\def\zbf{\ccvect{#3}[\rf]}
\def\origine{\ccent{#4}{#5}}
\def\originem{\ccentm{#4}{#5}}

\ifthenelse{\equal{\mode}{h}}{%

\texttt{\textbackslash fplanes[mode][a][a\_i][r\_t][col\_t][r\_f][col\_t]\{[axe1][axe2][axe3][c][c\_i]\}}\\

\begin{itemize}
    \item \texttt{mode} $=$ h, r, b (défault : \texttt{vide})
        \begin{itemize}
            \item \texttt{h} : afficher cette aide
            \item \texttt{vide} : figure plane
            \item \texttt{r} : figure plane $+$ vecteur rotation
            \item \texttt{b} : figure plane $+$ vecteur rotation $+$ vecteur vitesse
        \end{itemize}
    \item \texttt{a} : lettre grec de l'angle (défault : \texttt{alpha})
    \item \texttt{a\_i} : indice de l'angle (défault : \texttt{vide})
    \item \texttt{r\_t} : indice du repère tournant (défault : \texttt{1})
    \item \texttt{c\_t} : couleur du repère tournant (défault : \texttt{black})
    \item \texttt{r\_f} : indice du repère fixe (défault : \texttt{0})
    \item \texttt{c\_f} : couleur du repère fixe (défault : \texttt{black})
    \item \texttt{axe1} : l'axe pointant vers la droite (défault : \texttt{x})
    \item \texttt{axe2} : l'axe pointant vers le haut (défault : \texttt{y})
    \item \texttt{axe3} : l'axe sortant (défault : \texttt{z})
    \item \texttt{c} : origine des répères (défault : \texttt{O})
    \item \texttt{c\_i} : indice de l'origine (défault : \texttt{vide})
\end{itemize}
Remarques : 
\texttt{axe1,axe2,axe3} doivent être direct ! \texttt{xyz yzx zxy} !!
}{
    \coordinate (a) at (0,0);
    \coordinate (x0) at (2,0);
    \coordinate (y0) at (0,2);
    \coordinate (x1) at (1.8793852415718169,0.6840402866513374);
    \coordinate (y1) at (-0.6840402866513374,1.8793852415718169);
    \draw [-latex,thick,color=\cf] (a) -- (x0) node [right] {$\xbf$} ;
    \draw [-latex,thick,color=\ct] (a) -- (x1) node [right] {$\xbt$};
    \draw [-latex,thick,color=\cf] (a) -- (y0) node [above] {$\ybf$};
    \draw [-latex,thick,color=\ct] (a) -- (y1) node [above] {$\ybt$};
    \draw [-latex,thick] 
    pic["$\angle$",draw=black,-latex,
    angle eccentricity=1.2,angle radius=1.7cm] {angle=x0--a--x1}
    pic["$\angle$",draw=black,-latex,
    angle eccentricity=1.2,angle radius=1.7cm] {angle=y0--a--y1};
    \coordinate (z) at ($(a)+(-1,0.3)$);
    \draw [thick,fill=black] (a) circle (1pt)   node [below,yshift=-0.5ex]{\origine};
    \draw [thick] (z) circle (3pt)              node [\cf,below] {$\hphantom{\zbt}=\zbf$};
    \draw [thick,fill=black] (z) circle (0.5pt) node [\ct,below] {$\zbt\hphantom{=\zbf}$};
\ifthenelse{\equal{\mode}{r}}{%
    \node at (0.9,-0.8) {$\Om{(\rt/\rf)}=\dot{\angle}\zbf$};
}{}
\ifthenelse{\equal{\mode}{b}}{%
    \node at (0.9,-1.2) {$\begin{aligned}\Om{(\rt/\rf)}&=\dot{\angle}\zbf\\\TCV{}{\originem}{\rt}{\rf}&=\vnull
        \end{aligned}$};
}{}
}
}
\endinput
%%
%% End of file `fig-planes.sty'.
\end{verbatim}


\end{document}
