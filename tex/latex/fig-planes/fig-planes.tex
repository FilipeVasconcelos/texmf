\documentclass[a4paper,9pt]{article} 
\usepackage[utf8]{inputenc}          
\usepackage[T1]{fontenc}             
\usepackage[francais]{babel}         
\usepackage{lmodern}                 

\usepackage{geometry}                                                                                                         
\geometry{hmargin=2.5cm,vmargin=2.6cm}

\usepackage{mathfmv}
\usepackage{amsmath}
\usepackage{titling}
\usepackage{tikz,pgf}
\usepackage{fig-planes}
\usepackage{verbatim}

\newcommand{\subtitle}[1]{%                          
\posttitle{%                                   
\par\end{center}%
\begin{center}\large#1\end{center}%
\vskip0.5em}%              
}                          
\newcommand{\mparagraph}[1]{\paragraph{#1}\mbox{}\\} 
\title{Figures planes avec PGF/Tikz}
\subtitle{fig-planes version 2.5}
\author{F. M. Vasconcelos}
\date{}

\begin{document}
\maketitle
\begin{abstract}
Macros tikz pour tracer des figures planes pour la mécanique. Les figures planes
permettent de représenter des mouvements relatifs de rotation ou 
de translation entre deux repères.
L'ensemble de la figure est paramètrable jusqu'au choix des couleurs des repères 
tournant et fixe. La commande possède 14 options qui ont toutes des valeurs 
par défaut. 
\end{abstract}

%%%%%%%%%%%%%%%%%%%%%%%%%%%%%%%%%%%%%%%%%%%%%%%%%%%%%%%%%%%%%%%%%%%%%%%%%%%%%%%%%     
\section{Paquets nécessaires}                                                         
%%%%%%%%%%%%%%%%%%%%%%%%%%%%%%%%%%%%%%%%%%%%%%%%%%%%%%%%%%%%%%%%%%%%%%%%%%%%%%%%%     
\begin{verbatim}                                                                      
\usepackage{tikz}
\usetikzlibrary{calc,quotes,angles}
\RequirePackage{amsmath}
\RequirePackage{kinematik}
\usepackage{ifthen}
\end{verbatim} 
%%%%%%%%%%%%%%%%%%%%%%%%%%%%%%%%%%%%%%%%%%%%%%%%%%%%%%%%%%%%%%%%%%%%%%%%%%%%%%%%%     
\section{Utilisation des macros}                                                         
%%%%%%%%%%%%%%%%%%%%%%%%%%%%%%%%%%%%%%%%%%%%%%%%%%%%%%%%%%%%%%%%%%%%%%%%%%%%%%%%%     
\subsection*{Accès à l'aide du package}

\figplanes[h]

Commande de l'aide :
\begin{verbatim}
\figplanes[h]
\end{verbatim}
Remarque :
Les balises tikz \texttt{\textbackslash begin\{tikzpicture\}\ldots \textbackslash end\{tikzpicture\}} ne 
sont pas nécessaire dans le mode \texttt{h}

\subsection*{Figure par défaut}

\begin{center}
    \begin{tikzpicture}
        \figplanes
    \end{tikzpicture}
\end{center}

Commande:
\begin{verbatim}
\begin{tikzpicture}
\figplanes
\end{tikzpicture}
\end{verbatim}

\subsection*{Différents modes}

\paragraph{\texttt{mode=r}}$\,$

\begin{center}
    \begin{tikzpicture}
        \figplanes[r]
    \end{tikzpicture}
\end{center}

Commande:
\begin{verbatim}
\begin{tikzpicture}
\figplanes[r]
\end{tikzpicture}
\end{verbatim}

\paragraph{\texttt{mode=b}}$\,$

\begin{center}
    \begin{tikzpicture}
        \figplanes[b]
    \end{tikzpicture}
\end{center}

Commande:
\begin{verbatim}
\begin{tikzpicture}
\figplanes[b]
\end{tikzpicture}
\end{verbatim}


\paragraph{\texttt{mode=t}}$\,$

\begin{center}
    \begin{tikzpicture}
        \figplanes[t]
    \end{tikzpicture}
\end{center}

Commande:
\begin{verbatim}
\begin{tikzpicture}
\figplanes[t]
\end{tikzpicture}
\end{verbatim}

\paragraph{\texttt{mode=tv}}$\,$

\begin{center}
    \begin{tikzpicture}
        \figplanes[tv]
    \end{tikzpicture}
\end{center}

Commande:
\begin{verbatim}
\begin{tikzpicture}
\figplanes[tv]
\end{tikzpicture}
\end{verbatim}

\subsection*{Quelques exemples}

\begin{center}
    \begin{tikzpicture}
        \figplanes[r][theta][2][3][red][2][blue]{[x][y][z][A][6]}
    \end{tikzpicture}
\end{center}

Commande:
\begin{verbatim}
\begin{tikzpicture}
\figplanes[r][theta][2][3][red][2][blue]{[x][y][z][A][6]}
\end{tikzpicture}
\end{verbatim}

\begin{center}
    \begin{tikzpicture}
        \figplanes[b][beta][1][1][green!80][0][orange!80]{[y][z][x][B][4]}
    \end{tikzpicture}
\end{center}

Commande:
\begin{verbatim}
\begin{tikzpicture}
\figplanes[b][beta][1][1][green!80][0][orange!80]{[y][z][x][B][4]}
\end{tikzpicture}
\end{verbatim}

\begin{center}
    \begin{tikzpicture}
        \figplanes[t][lambda][1][3][red][2][blue]{[y][z][x][A][6][C][4]}
    \end{tikzpicture}
\end{center}

Commande:
\begin{verbatim}
\begin{tikzpicture}
    \figplanes[t][lambda][1][3][red][2][blue]{[y][z][x][A][6][C][4]}
\end{tikzpicture}
\end{verbatim}

\clearpage
\section{Code des macros}
\footnotesize
\begin{verbatim}
% fig-planes package
% 
% (c) Filipe Vasconcelos
%
%% This program can be redistributed and/or modified under the terms
%% of the LaTeX Project Public License Distributed from CTAN archives
%% in directory macros/latex/base/lppl.txt.
% 
\NeedsTeXFormat{LaTeX2e}[1994/06/01]
\ProvidesPackage{fig-planes}
  [2017/06/05 v0.01 LaTeX package dessiner des figures planes;
   2018/09/26 v0.02 LaTeX package dessiner des figures planes en couleur
   2019/12/12 v0.03 LaTeX package dessiner des figures planes en couleur (rotation et translation)]

\RequirePackage{xargs}
\RequirePackage{xparse}
\RequirePackage{ifthen}
\RequirePackage{mathfmv}
\RequirePackage{tikz}
\usetikzlibrary{calc,quotes,angles}
\RequirePackage{amsmath}
\RequirePackage{kinematik}
\def\hi{Ce paquet dessine des figures planes tikz de changement 
de base pour la cinématique.}
\let\myDate\date

\newcommandx{\figplanes}[8][1=,2=alpha,3=,4=1,5=black,6=0,7=black,usedefault]{ 
\def\mode{#1}
\ifthenelse{\equal{\mode}{t}\OR\equal{\mode}{tv}}{%
    \ifthenelse{\equal{#2}{alpha}}{%
        \def\angle{\cang{lambda}{#3}}
    }{\def\angle{\cang{#2}{#3}}}
}{
    \def\angle{\cang{#2}{#3}}
}
\def\rt{#4}
\def\ct{#5} % color
\def\rf{#6}
\def\cf{#7} % color
\fplanescontinue#8
}

\newcommandx{\fplanescontinue}[7][1=x,2=y,3=z,4=O,5=,6=O,7=1,usedefault]{ 
\def\xbt{\ccvect{#1}[\rt]}
\def\xbf{\ccvect{#1}[\rf]}
\def\ybt{\ccvect{#2}[\rt]}
\def\ybf{\ccvect{#2}[\rf]}
\def\zbt{\ccvect{#3}[\rt]}
\def\zbf{\ccvect{#3}[\rf]}
\def\origine{\ccent{#4}{#5}}
\def\originem{\ccentm{#4}{#5}}
\def\origined{\ccent{#6}{#7}}
\def\originedm{\ccentm{#6}{#7}}

\ifthenelse{\equal{\mode}{h}}{%
    %help mode
    \texttt{\textbackslash fplanes[mode][a][a\_i][r\_t][col\_t][r\_f][col\_t]\{[axe1][axe2][axe3][c][c\_i][t][t\_i]\}}\\

\begin{itemize}
    \item \texttt{mode} $=$ h, t, r, b (défault : \texttt{vide})
        \begin{itemize}
            \item \texttt{h}    : afficher cette aide
            \item \texttt{vide} : figure plane en rotation
            \item \texttt{r}    : figure plane $+$ vecteur rotation
            \item \texttt{b}    : figure plane $+$ vecteur rotation $+$ vecteur vitesse
            \item \texttt{t}    : figure plane en translation
            \item \texttt{tv}   : figure plane en translation $+$ vecteur vitesse
        \end{itemize}
    \item \texttt{a}    : lettre grec de l'angle/longueur (défault : \texttt{alpha} ou \texttt{lambda})
    \item \texttt{a\_i} : indice de l'angle (défault : \texttt{vide})
    \item \texttt{r\_t} : indice du repère tournant/translaté  (défault : \texttt{1})
    \item \texttt{c\_t} : couleur du repère tournant/translaté (défault : \texttt{black})
    \item \texttt{r\_f} : indice du repère fixe (défault : \texttt{0})
    \item \texttt{c\_f} : couleur du repère fixe (défault : \texttt{black})
    \item \texttt{axe1} : l'axe pointant vers la droite (défault : \texttt{x})
    \item \texttt{axe2} : l'axe pointant vers le haut (défault : \texttt{y})
    \item \texttt{axe3} : l'axe sortant (défault : \texttt{z})
    \item \texttt{c}    : origine des repères (défault : \texttt{O})
    \item \texttt{c\_i} : indice de l'origine (défault : \texttt{vide})
    \item \texttt{t}    : origine du  repère translaté (défault : \texttt{O})
    \item \texttt{t\_i} : indice de l'origine du repère translaté  (défault : \texttt{1})
\end{itemize}
Remarques : 
\texttt{axe1,axe2,axe3} doivent être direct ! \texttt{xyz yzx zxy} !!
}{%if not help mode
    \ifthenelse{\equal{\mode}{t}\OR\equal{\mode}{tv}}{
        % figures planes de rotation d'un repere par rapport à un autre
        \coordinate (a) at (0,0);
        \coordinate (b) at (1,0);
        \coordinate (x0) at (1,0);
        \coordinate (y0) at (0,1);
        \coordinate (x1) at (1,0);
        \coordinate (y1) at (0,1);
        \draw [-latex,thick,color=\ct] (b) -- ($(b)+2.4*(x1)$) node [right] {$\xbt$};
        \draw [-latex,thick,color=\cf] (a) -- ($(a)+2.4*(x0)$) node [below] {$\xbf$} ;
        \draw [-latex,thick,color=\cf] (a) -- ($(a)+2.4*(y0)$) node [above] {$\ybf$};
        \draw [-latex,thick,color=\ct] (b) -- ($(b)+2.4*(y1)$) node [above] {$\ybt$};
        \draw [thick,fill=black] (a) circle (1pt)   node [below,yshift=-0.5ex]{\origine};
        \draw [thick,fill=black] (b) circle (1pt)   node [below,yshift=-0.5ex]{\origined};
        \coordinate (z) at ($(a)+(-1,0.3)$);
        \draw [thick] (z) circle (3pt)              node [\cf,below] {$\hphantom{\zbt}=\zbf$};
        \draw [thick,fill=black] (z) circle (0.5pt) node [\ct,below] {$\zbt\hphantom{=\zbf}$};
        \draw [-latex,thick] ($(a)+1.7*(y0)$) -- node[above] {$\angle$} ($(b)+1.7*(y0)$);
        \ifthenelse{\equal{\mode}{tv}}{%
            \node at (0.9,-1.0) {$\TCV{\originedm}{\rt}{\rf}=\dot{\angle}\xbt$};
        }{}
    }{
        \coordinate (a) at (0,0);
        \coordinate (x0) at (1,0);
        \coordinate (y0) at (0,1);
        \coordinate (x1) at (0.93969256961929659, 0.34202028390474665);
        \coordinate (y1) at (-0.34202028390474665, 0.93969256961929659);

        \draw [-latex,thick,color=\cf] (a) -- ($(a)+2.4*(x0)$) node [right] {$\xbf$} ;
        \draw [-latex,thick,color=\ct] (a) -- ($(a)+2.4*(x1)$) node [right] {$\xbt$};
        \draw [-latex,thick,color=\cf] (a) -- ($(a)+2.4*(y0)$) node [above] {$\ybf$};
        \draw [-latex,thick,color=\ct] (a) -- ($(a)+2.4*(y1)$) node [above] {$\ybt$};
        \draw [-latex,thick] 
        pic["$\angle$",draw=black,-latex,
        angle eccentricity=1.2,angle radius=1.7cm] {angle=x0--a--x1}
        pic["$\angle$",draw=black,-latex,
        angle eccentricity=1.2,angle radius=1.7cm] {angle=y0--a--y1};
        \coordinate (z) at ($(a)+(-1,0.3)$);
        \draw [thick,fill=black] (a) circle (1pt)   node [below,yshift=-0.5ex]{\origine};
        \draw [thick] (z) circle (3pt)              node [\cf,below] {$\hphantom{\zbt}=\zbf$};
        \draw [thick,fill=black] (z) circle (0.5pt) node [\ct,below] {$\zbt\hphantom{=\zbf}$};
        % if mode not t but r or b
        \ifthenelse{\equal{\mode}{r}}{%
            \node at (0.9,-1.0) {$\Om{\rt/\rf}=\dot{\angle}\zbf$};}
        {}
        \ifthenelse{\equal{\mode}{b}}{%
        \node at (0.9,-1.2) {$\begin{aligned}\Om{\rt/\rf}&=\dot{\angle}\zbf\\\TCV{\originem}{\rt}{\rf}&=\vnull\end{aligned}$};
        }{}
    }
}

}
\endinput
%%
%% End of file `fig-planes.sty'.
\end{verbatim}


\end{document}
