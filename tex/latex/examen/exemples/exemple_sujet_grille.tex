\documentclass[11pt,a4paper,twoside]{article} %twocolumn
\usepackage[francais]{babel}
\usepackage[babel=true,kerning=true]{microtype}
\usepackage[utf8]{inputenc}
\usepackage[T1]{fontenc}
\usepackage{lmodern}
\usepackage{amsmath,amssymb}
\usepackage[top=2cm,bottom=2.2cm,right=2.2cm,left=2.2cm]{geometry}
\usepackage{graphicx}
\graphicspath{{./img/}}
\usepackage{examen}
%%%%%%%%%%%%%%%%%%%%%%%%%%%%%%%%%%%%%%%%%%%%%%%%%%%%%%%%%%%%%%%%%%%%%%%%%%%%%%%%
%Modifier les variables suivantes 
\promo{IngéSUP}                       % ex. IngéSUP, IngéSPE, Ingé1
\module{Mathématiques Fondamentales}  % ex. Systèmes Techniques, Mathématiques Fondamentales
\annee{2024-2025}                     % ex. 2022-2023
\epreuve{MidTerm}                     % ex. MidTerm, FinalExam, Rattrapage
\titreEval{Algèbre Linéaire}          % ex. Dynamique et Déformation
\dureeEval{2}                         % ex. 2 (en nombre d'heures)
\esme{true}                           % examen ESME true or false
\documentautorise{false}                               % les documents sont-ils autorisés true or false 
\moyencalcul{false}                                    % les moyens de calculs sont-ils autorisés
\dispositions{true}                                    % Remarque en cas de découverte de coquille
\grille{true}                         % édition énoncé du document avec grille réponse true or false
\corrige{false}                       % edition corrigé du document true or false
%%%%%%%%%%%%%%%%%%%%%%%%%%%%%%%%%%%%%%%%%%%%%%%%%%%%%%%%%%%%%%%%%%%%%%%%%%%%%%%%
%%%%%%%%%%%%%%%%%%%%%%%%%%%%%%%%%%%%%%%%%%%%%%%%%%%%%%%%%%%%%%%%%%%%%%%%%%%%%%%%
\pagestyle{esmestyle}
\begin{document}
\maketitle
%%%%%%%%%%%%%%%%%%%%%%%%%%%%%%%%%%%%%%%%%%%%%%%%%%%%%%%%%%%%%%%%%%%%%%%%%%%%%%%%
\thispagestyle{esmestyle}
%%%%%%%%%%%%%%%%%%%%%%%%%%%%%%%%%%%%%%%%%%%%%%%%%%%%%%%%%%%%%%%%%%%%%%%%%%%%%%%%
\exercice{Diagonalisation d'une matrice 3x3}
%%%%%%%%%%%%%%%%%%%%%%%%%%%%%%%%%%%%%%%%%%%%%%%%%%%%%%%%%%%%%%%%%%%%%%%%%%%%%%%%
On considère la matrice :
\[
A = \begin{bmatrix} 1 & 2 & 0 \\ 0 & 3 & 0 \\ 2 & -4 & 2 \end{bmatrix}
\]
\question{Déterminer une matrice diagonale \(D\) et une matrice inversible \(P\) telles que :}
\[
A = P D P^{-1}
\]
\grillereponse[20cm]{
\subsection*{Étape 1 : Calcul du polynôme caractéristique}
Le polynôme caractéristique est donné par :
\[
\det(A - \lambda I) = \begin{vmatrix} 1 - \lambda & 2 & 0 \\ 0 & 3 - \lambda & 0 \\ 2 & -4 & 2 - \lambda \end{vmatrix}
\]

En développant selon la dernière colonne :
\[
(2-\lambda) \begin{vmatrix} 1-\lambda & 2 \\ 0 & 3-\lambda \end{vmatrix}
\]

Le déterminant du mineur \( 2 \times 2 \) est :
\[
    (1-\lambda)(3-\lambda)
\]

Ainsi, l'équation caractéristique est :
\[
(1-\lambda)(2-\lambda)(3-\lambda) = 0
\]

\subsection*{Étape 2 : Détermination des valeurs propres et vecteurs propres}
Les solutions de \(1-\lambda)(2-\lambda)(3-\lambda) = 0\) sont :
\[
\lambda_1 = 1, \quad \lambda_2 = 2, \quad \lambda_3 = 3
\]

Les vecteurs propres associés sont :
\[
    v_1=\begin{pmatrix}1\\0\\-2\end{pmatrix}\;
    v_2=\begin{pmatrix}0\\0\\1\end{pmatrix}\;
    v_3=\begin{pmatrix}1\\1\\-2\end{pmatrix}\;
\]

On construit la matrice \( P \) dont 
les colonnes sont ces vecteurs propres, et la matrice diagonale \( D \) est :

\[
P = \begin{bmatrix} 1 & 0 & 1 \\ 0 & 0 & 1 \\ -2 & 1 & -2 \end{bmatrix}
\]
\[
D = \begin{bmatrix} 1 & 0 & 0 \\ 0 & 2 & 0 \\ 0 & 0 & 3 \end{bmatrix}
\]
}
\clearpage
%%%%%%%%%%%%%%%%%%%%%%%%%%%%%%%%%%%%%%%%%%%%%%%%%%%%%%%%%%%%%%%%%%%%%%%%%%%%%%%%
%%%%%%%%%%%%%%%%%%%%%%%%%%%%%%%%%%%%%%%%%%%%%%%%%%%%%%%%%%%%%%%%%%%%%%%%%%%%%%%%
\exercice{Opérations sur les polynômes}
%%%%%%%%%%%%%%%%%%%%%%%%%%%%%%%%%%%%%%%%%%%%%%%%%%%%%%%%%%%%%%%%%%%%%%%%%%%%%%%%
Soient \( a, b \) des réels, et considérons le polynôme :
\[
P(X) = X^4 + 2aX^3 + bX^2 + 2X + 1.
\]
\question{Déterminer les valeurs de \( a \) et \( b \) pour lesquelles \( P(X) \) 
          est le carré d'un polynôme à coefficients réels, c'est-à-dire 
          qu'il existe un polynôme \( Q(X) \in \mathbb{R}[X] \) tel que :}
\[
P(X) = (Q(X))^2.
\]
\grillereponse[19cm]{%
Si \( P = Q^2 \) est le carré d'un polynôme, alors \( Q \) est nécessairement 
de degré 2, et son coefficient dominant est égal à \( 1 \) ou à \( -1 \).
Dans le premier cas, on peut donc écrire :
\[
Q(X) = X^2 + cX + d.
\]
On a alors :
\[
Q^2(X) = X^4 + 2cX^3 + (2d + c^2)X^2 + 2cdX + d^2.
\]

Par identification avec \( P(X) = X^4 + 2aX^3 + bX^2 + 2X + 1 \), 
on obtient le système :
\[
2c = 2a, \quad 2d + c^2 = b, \quad 2cd = 2, \quad d^2 = 1.
\]

On en déduit que \( c = a \) et \( d = \pm1 \).
- Si \( d = 1 \), alors \( c = 1 \), donc \( a = 1 \) et \( b = 3 \).
- Si \( d = -1 \), alors \( c = -1 \), donc \( a = -1 \) et \( b = -1 \).

Les deux solutions sont donc :
\[
P_1(X) = X^4 + 2X^3 + 3X^2 + 2X + 1 = (X^2 + X + 1)^2,
\]
\[
P_2(X) = X^4 - 2X^3 - X^2 + 2X + 1 = (X^2 - X - 1)^2.
\]

Dans le deuxième cas, on écrit \( Q(X) = -R(X) \) avec 
\( R(X) = X^2 + cX + d \), de sorte que :
\[
Q^2(X) = R^2(X).
\]
On retrouve alors en réalité le cas précédent.
}
\end{document}
