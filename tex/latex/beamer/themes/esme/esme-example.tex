\documentclass[aspectratio=169]{beamer}
\usepackage[french]{babel}
\usepackage[utf8]{inputenc}
\usepackage[T1]{fontenc}
\usepackage{lmodern}
\usepackage{verbatim}
\title{Modèle de diaporama ESME}
\subtitle{Sous titre}
\date[]{}
\author[]{}

\usetheme{esme}

\begin{document}

%%%%%%%%%%%%%%%%%%%%%%%%%%%%%%%%%%%%%%%%%%%%%%%%%%%%%%%%%%%%%%%%%%%%%%%%%%%%%%%%
\section{Introduction}
%%%%%%%%%%%%%%%%%%%%%%%%%%%%%%%%%%%%%%%%%%%%%%%%%%%%%%%%%%%%%%%%%%%%%%%%%%%%%%%%

%%%%%%%%%%%%%%%%%%%%%%%%%%%%%%%%%%%%%%%%%%%%%%%%%%%%%%%%%%%%%%%%%%%%%%%%%%%%%%%%
\begin{frame}[fragile] 
\frametitle{Introduction} 
\framesubtitle{Sous titre de la frame} 

Nous avons mis au point un thème \texttt{beamer} \og ESME\fg~pour 
\LaTeX~à destination des concepteurs de ressources pédagogiques 
numériques.\newline

Ce modèle prend la forme d'un \og template\fg~contenant des 
dispositions types.\newline

Ce modèle est conforme à la charte graphique de l'ESME (couleurs, 
logos).

\emph{N.B : la police d'écriture est celle de~\LaTeX}\newline

\begin{verbatim}
\usepackage[T1]{fontenc}
\texttt{lmodern}
\end{verbatim}

Les fonctionnalités de beamer sont conservées.
\end{frame}
%%%%%%%%%%%%%%%%%%%%%%%%%%%%%%%%%%%%%%%%%%%%%%%%%%%%%%%%%%%%%%%%%%%%%%%%%%%%%%%%

%%%%%%%%%%%%%%%%%%%%%%%%%%%%%%%%%%%%%%%%%%%%%%%%%%%%%%%%%%%%%%%%%%%%%%%%%%%%%%%%
\begin{frame}{Sommaire}
    \tableofcontents
    \scriptsize
    Utilisez les commandes \textbackslash section, \textbackslash subsection ...
    tout le long du document et obtenez automatiquement une page Sommaire 
    comme celle-ci à l'appel de la section. 
\end{frame}
%%%%%%%%%%%%%%%%%%%%%%%%%%%%%%%%%%%%%%%%%%%%%%%%%%%%%%%%%%%%%%%%%%%%%%%%%%%%%%%%

%%%%%%%%%%%%%%%%%%%%%%%%%%%%%%%%%%%%%%%%%%%%%%%%%%%%%%%%%%%%%%%%%%%%%%%%%%%%%%%%
\begin{frame} 
    \frametitle{Exemples de \texttt{\textbackslash sectionframe}}

    Vous pouvez utiliser (actuellement) 3 types de pages de titre pour
    les sections avec la commande \texttt{\textbackslash sectionframe\{n\}} 
    où $n=1,2,3$.\newline

    Cette page indiquera le numéro et le titre de la section dans laquelle
    vous vous trouvez.\newline

    Les trois \texttt{frame} suivantes montrent 3 exemples.
\end{frame}
%%%%%%%%%%%%%%%%%%%%%%%%%%%%%%%%%%%%%%%%%%%%%%%%%%%%%%%%%%%%%%%%%%%%%%%%%%%%%%%%
% 3 exemples de sectionframe
\sectionframe{1}
\sectionframe{2}
\sectionframe{3}
%%%%%%%%%%%%%%%%%%%%%%%%%%%%%%%%%%%%%%%%%%%%%%%%%%%%%%%%%%%%%%%%%%%%%%%%%%%%%%%%

\section{Exemples d'environnement beamer}
\sectionframe{3}

%%%%%%%%%%%%%%%%%%%%%%%%%%%%%%%%%%%%%%%%%%%%%%%%%%%%%%%%%%%%%%%%%%%%%%%%%%%%%%%%
\begin{frame}[fragile]
    \frametitle{Exemple 1: Environnement \texttt{itemize}}
\begin{itemize}
    \item premier élément 
        \begin{itemize}
            \item Niveau 2
        \end{itemize}
    \item second élément
\end{itemize}
\begin{verbatim}
    \begin{itemize}
    \item premier élément 
    \item second élément
    \end{itemize}
\end{verbatim}
\end{frame}
%%%%%%%%%%%%%%%%%%%%%%%%%%%%%%%%%%%%%%%%%%%%%%%%%%%%%%%%%%%%%%%%%%%%%%%%%%%%%%%%

%%%%%%%%%%%%%%%%%%%%%%%%%%%%%%%%%%%%%%%%%%%%%%%%%%%%%%%%%%%%%%%%%%%%%%%%%%%%%%%%
\begin{frame}[fragile]
    \frametitle{Exemple 2: Environnement \texttt{column}}
    \begin{columns}
        \column{0.75\linewidth}
        \begin{itemize}
            \item Avec contenu plus large à gauche
        \end{itemize}
        \column{0.25\linewidth}
        \begin{itemize}
            \item Et contenu plus étroit à droite 
        \end{itemize}
    \end{columns}
{\scriptsize
\begin{verbatim}
    \begin{columns}
        \column{0.75\linewidth}
        \begin{itemize}
            \item Avec contenu plus large à gauche
        \end{itemize}
        \column{0.25\linewidth}
        \begin{itemize}
            \item Et contenu plus étroit à droite 
        \end{itemize}
    \end{columns}
\end{verbatim}
}
\end{frame}
%%%%%%%%%%%%%%%%%%%%%%%%%%%%%%%%%%%%%%%%%%%%%%%%%%%%%%%%%%%%%%%%%%%%%%%%%%%%%%%%

\section{Différentes \texttt{frametitle}}
\sectionframe{3}
%%%%%%%%%%%%%%%%%%%%%%%%%%%%%%%%%%%%%%%%%%%%%%%%%%%%%%%%%%%%%%%%%%%%%%%%%%%%%%%%
\begin{frame}[question]
    \frametitle{Exemple \og Question\fg}

Pour utiliser une frame question (c.f icône à droite), taper la commande :
\noindent\texttt{%
\textbackslash begin\{frame\}[question]\\
\ldots\emph{Votre slide}\ldots\\
\textbackslash end\{frame\}
}
\end{frame}
%%%%%%%%%%%%%%%%%%%%%%%%%%%%%%%%%%%%%%%%%%%%%%%%%%%%%%%%%%%%%%%%%%%%%%%%%%%%%%%%

%%%%%%%%%%%%%%%%%%%%%%%%%%%%%%%%%%%%%%%%%%%%%%%%%%%%%%%%%%%%%%%%%%%%%%%%%%%%%%%%
\begin{frame}[reponse]
    \frametitle{Exemple \og Réponse\fg}

Pour utiliser une frame question (c.f icône à droite), taper la commande :
\noindent\texttt{%
\textbackslash begin\{frame\}[reponse]\\
\ldots\emph{Votre slide}\ldots\\
\textbackslash end\{frame\}
}
\end{frame}
%%%%%%%%%%%%%%%%%%%%%%%%%%%%%%%%%%%%%%%%%%%%%%%%%%%%%%%%%%%%%%%%%%%%%%%%%%%%%%%%
\section{Les blocs}
\sectionframe{3}
%%%%%%%%%%%%%%%%%%%%%%%%%%%%%%%%%%%%%%%%%%%%%%%%%%%%%%%%%%%%%%%%%%%%%%%%%%%%%%%%
\begin{frame}
\frametitle{Les \og blocs\fg}
\begin{block}{Bloc générique}
L'environnement \texttt{block} de beamer permet de faire des blocs
autour d'un texte
\end{block}
\noindent\texttt{%
\textbackslash begin\{block\}\{Bloc générique\}\\
\ldots\emph{Votre texte}\ldots\\
\textbackslash end\{block\}
}%
\end{frame}
%%%%%%%%%%%%%%%%%%%%%%%%%%%%%%%%%%%%%%%%%%%%%%%%%%%%%%%%%%%%%%%%%%%%%%%%%%%%%%%%
%%%%%%%%%%%%%%%%%%%%%%%%%%%%%%%%%%%%%%%%%%%%%%%%%%%%%%%%%%%%%%%%%%%%%%%%%%%%%%%%
\begin{frame}
\frametitle{Bloc \og Définition\fg}
\begin{blockDef}
Changer le texte ici
\end{blockDef}
\noindent\texttt{%
\textbackslash begin\{blockDef\}\\
\ldots\emph{Votre texte}\ldots\\
\textbackslash end\{blockDef\}
}%
\end{frame}
%%%%%%%%%%%%%%%%%%%%%%%%%%%%%%%%%%%%%%%%%%%%%%%%%%%%%%%%%%%%%%%%%%%%%%%%%%%%%%%%
%%%%%%%%%%%%%%%%%%%%%%%%%%%%%%%%%%%%%%%%%%%%%%%%%%%%%%%%%%%%%%%%%%%%%%%%%%%%%%%%
\begin{frame}
    \frametitle{Bloc \og Propriétés\fg}
    \begin{blockProp}
        Changer le texte ici
    \end{blockProp}
\noindent\texttt{%
\textbackslash begin\{blockProp\}\\
\ldots\emph{Votre texte}\ldots\\
\textbackslash end\{blockProp\}
}%
\end{frame}
%%%%%%%%%%%%%%%%%%%%%%%%%%%%%%%%%%%%%%%%%%%%%%%%%%%%%%%%%%%%%%%%%%%%%%%%%%%%%%%%
%%%%%%%%%%%%%%%%%%%%%%%%%%%%%%%%%%%%%%%%%%%%%%%%%%%%%%%%%%%%%%%%%%%%%%%%%%%%%%%%
\begin{frame}
\frametitle{Bloc \og Exemple\fg}
\begin{blockEx}
Changer le texte ici
\end{blockEx}
\noindent\texttt{%
\textbackslash begin\{blockEx\}\\
\ldots\emph{Votre texte}\ldots\\
\textbackslash end\{blockEx\}
}%
\end{frame}
%%%%%%%%%%%%%%%%%%%%%%%%%%%%%%%%%%%%%%%%%%%%%%%%%%%%%%%%%%%%%%%%%%%%%%%%%%%%%%%%
%%%%%%%%%%%%%%%%%%%%%%%%%%%%%%%%%%%%%%%%%%%%%%%%%%%%%%%%%%%%%%%%%%%%%%%%%%%%%%%%
\begin{frame}
\frametitle{Bloc \og Remarque\fg}
\begin{blockRem}
Changer le texte ici
\end{blockRem}
\noindent\texttt{%
\textbackslash begin\{blockRem\}\\
\ldots\emph{Votre texte}\ldots\\
\textbackslash end\{blockRem\}}%
\end{frame}
%%%%%%%%%%%%%%%%%%%%%%%%%%%%%%%%%%%%%%%%%%%%%%%%%%%%%%%%%%%%%%%%%%%%%%%%%%%%%%%%


\end{document}
