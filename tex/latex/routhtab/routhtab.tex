\documentclass[a4paper,9pt]{article}
\usepackage[utf8]{inputenc}
\usepackage[T1]{fontenc}
\usepackage[francais]{babel}
\usepackage{lmodern}
\usepackage{titling}

\usepackage{amsmath}
\usepackage{pgf,tikz,pgfplots}
\usepackage{tikzfmv}
\usetikzlibrary{external}
\tikzset{external/system call={pdflatex -shell-escape
-halt-on-error -interaction=batchmode -jobname "\image" "\texsource";
pdf2eps 1 "\image"
}}
\tikzexternalize[prefix=./figtikz/]
\usetikzlibrary{external}
\usepackage{routhtab}

\newcommand{\subtitle}[1]{%                          
\posttitle{%                                   
\par\end{center}%
\begin{center}\large#1\end{center}%
\vskip0.5em}%              
}

\title{Macros pour la création de tableau de Routh}
\subtitle{\texttt{routhtab} version 1.0}
\author{F. M. Vasconcelos}
\date{}

\begin{document}
\maketitle
\begin{abstract}
    Macros pour la création de tableau de Routh.
\end{abstract}

\section{Usage}

\tikzsetnextfilename{routh-test}
\begin{tikzpicture}
    \draw[ultra thick] (0,0) -- (10,10);
\end{tikzpicture}


\[
\begin{matrix}
    p^3 \\
    p^2 \\
    p^1 \\
    p^0 \\
\end{matrix}
\begin{vmatrix}
    1\DoTikzmark{-1ex}{1ex}{n1} & 3\DoTikzmark{1ex}{1ex}{n3}  \\
    1\DoTikzmark{-1ex}{-6ex}{n2}     & K  \\
    3-K                              & 0\DoTikzmark{1ex}{-3.5ex}{n4}  \\
    K                                & 0    
    \end{vmatrix}
\]
\colrow[green,opacity=.2]{n1}{n2}
\colrow[col1,opacity=.2]{n3}{n4}[40]

\[
    \begin{matrix}
            p^n    \\
            p^{n-1}\\
            p^{n-2}\\
            p^{n-3}\\
            \vdots \\
            p^1    \\
            p^0    \\
    \end{matrix}
    \begin{vmatrix}
            b_n\DoTikzmark{-1.1ex}{1.75ex}{n1}  & b_{n-2}    & b_{n-4}    & \cdots & b_2        & b_0         \\
            b_{n-1}                 & b_{n-3}    & b_{n-5}    & \cdots & b_1        & 0           \\
            A_{31}                  & A_{32}     & A_{33}     & \cdots & A_{3(k-1)} & 0           \\
            A_{41}                  & A_{42}     & A_{43}     & \cdots & 0          & 0           \\
            \vdots                  & \vdots     & \vdots     & \vdots & \vdots     & 0           \\
            A_{(n-1)1}              & A_{(n-1)2} & 0          & \cdots & 0          & 0           \\
            b_0\DoTikzmark{-1.1ex}{-1ex}{n2}  & 0          & 0          & \cdots & 0          & 0
    \end{vmatrix}
    \]
\colrow[green,opacity=.2]{n1}{n2}

\end{document}
